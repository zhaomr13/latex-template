% THis file contains all the default packages and modifications for
% LHCb formatting

%% %%%%%%%%%%%%%%%%%%
%%  Page formatting
%% %%%%%%%%%%%%%%%%%%
%%\usepackage[margin=1in]{geometry}
\usepackage[top=1in, bottom=1.25in, left=1in, right=1in]{geometry}

% fallback for manual settings... uncomment if the geometry package is not available
%
%\voffset=-11mm
%\textheight=220mm
%\textwidth=160mm
%\oddsidemargin=0mm
%\evensidemargin=0mm

\columnsep=5mm
\addtolength{\belowcaptionskip}{0.5em}

\renewcommand{\textfraction}{0.01}
\renewcommand{\floatpagefraction}{0.99}
\renewcommand{\topfraction}{0.9}
\renewcommand{\bottomfraction}{0.9}

% Allow the page size to vary a bit ...
\raggedbottom
% To avoid Latex to be too fussy with line breaking ...
\sloppy

%% %%%%%%%%%%%%%%%%%%%%%%%
%% Packages to be used
%% %%%%%%%%%%%%%%%%%%%%%%% 
\usepackage{microtype}
\usepackage{lineno}  % for line numbering during review
\usepackage{xspace} % To avoid problems with missing or double spaces after
                    % predefined symbold
\usepackage{caption} %these three command get the figure and table captions automatically small
\renewcommand{\captionfont}{\small}
\renewcommand{\captionlabelfont}{\small}

%% Graphics
\usepackage{graphicx}  % to include figures (can also use other packages)
\usepackage{color}
\usepackage{colortbl}
\graphicspath{{./figs/}} % Make Latex search fig subdir for figures

%% Math
\usepackage{amsmath} % Adds a large collection of math symbols
\usepackage{amssymb}
\usepackage{amsfonts}
\usepackage{upgreek} % Adds in support for greek letters in roman typeset

%% fix to allow peaceful coexistence of line numbering and
%% mathematical objects
%% http://www.latex-community.org/forum/viewtopic.php?f=5&t=163
%%
\newcommand*\patchAmsMathEnvironmentForLineno[1]{%
\expandafter\let\csname old#1\expandafter\endcsname\csname #1\endcsname
\expandafter\let\csname oldend#1\expandafter\endcsname\csname
end#1\endcsname
 \renewenvironment{#1}%
   {\linenomath\csname old#1\endcsname}%
   {\csname oldend#1\endcsname\endlinenomath}%
}
\newcommand*\patchBothAmsMathEnvironmentsForLineno[1]{%
  \patchAmsMathEnvironmentForLineno{#1}%
  \patchAmsMathEnvironmentForLineno{#1*}%
}
\AtBeginDocument{%
\patchBothAmsMathEnvironmentsForLineno{equation}%
\patchBothAmsMathEnvironmentsForLineno{align}%
\patchBothAmsMathEnvironmentsForLineno{flalign}%
\patchBothAmsMathEnvironmentsForLineno{alignat}%
\patchBothAmsMathEnvironmentsForLineno{gather}%
\patchBothAmsMathEnvironmentsForLineno{multline}%
\patchBothAmsMathEnvironmentsForLineno{eqnarray}%
}

% Get hyperlinks to captions and in references.
% These do not work with revtex. Use "hypertext" as class option instead.
\usepackage{hyperref}    % Hyperlinks in references
\usepackage[all]{hypcap} % Internal hyperlinks to floats.

%%% $Id: lhcb-symbols-def.tex 78711 2015-08-06 07:54:32Z apuignav $
%%% ======================================================================
%%% Purpose: Standard LHCb aliases
%%% Author: Originally Ulrik Egede, adapted by Tomasz Skwarnicki for templates,
%%% rewritten by Chris Parkes
%%% Maintainer : Ulrik Egede (2010 - 2012)
%%% Maintainer : Rolf Oldeman (2012 - 2014)
%%% =======================================================================

%%% To use this file outside the normal LHCb document environment, the
%%% following should be added in a preamble (before \begin{document}
%%%
%%%\usepackage{ifthen} 
%%%\newboolean{uprightparticles}
%%%\setboolean{uprightparticles}{false} %Set true for upright particle symbols
\usepackage{xspace} 
\usepackage{upgreek}

%%%%%%%%%%%%%%%%%%%%%%%%%%%%%%%%%%%%%%%%%%%%%%%%%%%%%%%%%%%%
%%%
%%% The following is to ensure that the template automatically can process
%%% this file.
%%%
%%% Add comments with at least three %%% preceding.
%%% Add new sections with one % preceding
%%% Add new subsections with two %% preceding
%%%%%%%%%%%%%%%%%%%%%%%%%%%%%%%%%%%%%%%%%%%%%%%%%%%%%%%%%%%%

%%%%%%%%%%%%%
% Experiments
%%%%%%%%%%%%%
\def\lhcb {\mbox{LHCb}\xspace}
\def\atlas  {\mbox{ATLAS}\xspace}
\def\cms    {\mbox{CMS}\xspace}
\def\alice  {\mbox{ALICE}\xspace}
\def\babar  {\mbox{BaBar}\xspace}
\def\belle  {\mbox{Belle}\xspace}
\def\cleo   {\mbox{CLEO}\xspace}
\def\cdf    {\mbox{CDF}\xspace}
\def\dzero  {\mbox{D0}\xspace}
\def\aleph  {\mbox{ALEPH}\xspace}
\def\delphi {\mbox{DELPHI}\xspace}
\def\opal   {\mbox{OPAL}\xspace}
\def\lthree {\mbox{L3}\xspace}
\def\sld    {\mbox{SLD}\xspace}
%%%\def\argus  {\mbox{ARGUS}\xspace}
%%%\def\uaone  {\mbox{UA1}\xspace}
%%%\def\uatwo  {\mbox{UA2}\xspace}
%%%\def\ux85 {\mbox{UX85}\xspace}
\def\cern {\mbox{CERN}\xspace}
\def\lhc    {\mbox{LHC}\xspace}
\def\lep    {\mbox{LEP}\xspace}
\def\tevatron {Tevatron\xspace}

%% LHCb sub-detectors and sub-systems

%%%\def\pu     {PU\xspace}
\def\velo   {VELO\xspace}
\def\rich   {RICH\xspace}
\def\richone {RICH1\xspace}
\def\richtwo {RICH2\xspace}
\def\ttracker {TT\xspace}
\def\intr   {IT\xspace}
\def\st     {ST\xspace}
\def\ot     {OT\xspace}
%%%\def\Tone   {T1\xspace}
%%%\def\Ttwo   {T2\xspace}
%%%\def\Tthree {T3\xspace}
%%%\def\Mone   {M1\xspace}
%%%\def\Mtwo   {M2\xspace}
%%%\def\Mthree {M3\xspace}
%%%\def\Mfour  {M4\xspace}
%%%\def\Mfive  {M5\xspace}
\def\spd    {SPD\xspace}
\def\presh  {PS\xspace}
\def\ecal   {ECAL\xspace}
\def\hcal   {HCAL\xspace}
%%%\def\bcm    {BCM\xspace}
\def\MagUp {\mbox{\em Mag\kern -0.05em Up}\xspace}
\def\MagDown {\mbox{\em MagDown}\xspace}

\def\ode    {ODE\xspace}
\def\daq    {DAQ\xspace}
\def\tfc    {TFC\xspace}
\def\ecs    {ECS\xspace}
\def\lone   {L0\xspace}
\def\hlt    {HLT\xspace}
\def\hltone {HLT1\xspace}
\def\hlttwo {HLT2\xspace}

%%% Upright (not slanted) Particles

\ifthenelse{\boolean{uprightparticles}}%
{\def\Palpha      {\ensuremath{\upalpha}\xspace}
 \def\Pbeta       {\ensuremath{\upbeta}\xspace}
 \def\Pgamma      {\ensuremath{\upgamma}\xspace}                 
 \def\Pdelta      {\ensuremath{\updelta}\xspace}                 
 \def\Pepsilon    {\ensuremath{\upepsilon}\xspace}                 
 \def\Pvarepsilon {\ensuremath{\upvarepsilon}\xspace}                 
 \def\Pzeta       {\ensuremath{\upzeta}\xspace}                 
 \def\Peta        {\ensuremath{\upeta}\xspace}                 
 \def\Ptheta      {\ensuremath{\uptheta}\xspace}                 
 \def\Pvartheta   {\ensuremath{\upvartheta}\xspace}                 
 \def\Piota       {\ensuremath{\upiota}\xspace}                 
 \def\Pkappa      {\ensuremath{\upkappa}\xspace}                 
 \def\Plambda     {\ensuremath{\uplambda}\xspace}                 
 \def\Pmu         {\ensuremath{\upmu}\xspace}                 
 \def\Pnu         {\ensuremath{\upnu}\xspace}                 
 \def\Pxi         {\ensuremath{\upxi}\xspace}                 
 \def\Ppi         {\ensuremath{\uppi}\xspace}                 
 \def\Pvarpi      {\ensuremath{\upvarpi}\xspace}                 
 \def\Prho        {\ensuremath{\uprho}\xspace}                 
 \def\Pvarrho     {\ensuremath{\upvarrho}\xspace}                 
 \def\Ptau        {\ensuremath{\uptau}\xspace}                 
 \def\Pupsilon    {\ensuremath{\upupsilon}\xspace}                 
 \def\Pphi        {\ensuremath{\upphi}\xspace}                 
 \def\Pvarphi     {\ensuremath{\upvarphi}\xspace}                 
 \def\Pchi        {\ensuremath{\upchi}\xspace}                 
 \def\Ppsi        {\ensuremath{\uppsi}\xspace}                 
 \def\Pomega      {\ensuremath{\upomega}\xspace}                 

 \def\PDelta      {\ensuremath{\Delta}\xspace}                 
 \def\PXi      {\ensuremath{\Xi}\xspace}                 
 \def\PLambda      {\ensuremath{\Lambda}\xspace}                 
 \def\PSigma      {\ensuremath{\Sigma}\xspace}                 
 \def\POmega      {\ensuremath{\Omega}\xspace}                 
 \def\PUpsilon      {\ensuremath{\Upsilon}\xspace}                 
 
 %\mathchardef\Deltares="7101
 %\mathchardef\Xi="7104
 %\mathchardef\Lambda="7103
 %\mathchardef\Sigma="7106
 %\mathchardef\Omega="710A


 \def\PA      {\ensuremath{\mathrm{A}}\xspace}                 
 \def\PB      {\ensuremath{\mathrm{B}}\xspace}                 
 \def\PC      {\ensuremath{\mathrm{C}}\xspace}                 
 \def\PD      {\ensuremath{\mathrm{D}}\xspace}                 
 \def\PE      {\ensuremath{\mathrm{E}}\xspace}                 
 \def\PF      {\ensuremath{\mathrm{F}}\xspace}                 
 \def\PG      {\ensuremath{\mathrm{G}}\xspace}                 
 \def\PH      {\ensuremath{\mathrm{H}}\xspace}                 
% \def\PI      {\ensuremath{\mathrm{I}}\xspace}                 
 \def\PJ      {\ensuremath{\mathrm{J}}\xspace}                 
 \def\PK      {\ensuremath{\mathrm{K}}\xspace}                 
 \def\PL      {\ensuremath{\mathrm{L}}\xspace}                 
 \def\PM      {\ensuremath{\mathrm{M}}\xspace}                 
 \def\PN      {\ensuremath{\mathrm{N}}\xspace}                 
 \def\PO      {\ensuremath{\mathrm{O}}\xspace}                 
 \def\PP      {\ensuremath{\mathrm{P}}\xspace}                 
 \def\PQ      {\ensuremath{\mathrm{Q}}\xspace}                 
 \def\PR      {\ensuremath{\mathrm{R}}\xspace}                 
 \def\PS      {\ensuremath{\mathrm{S}}\xspace}                 
 \def\PT      {\ensuremath{\mathrm{T}}\xspace}                 
 \def\PU      {\ensuremath{\mathrm{U}}\xspace}                 
 \def\PV      {\ensuremath{\mathrm{V}}\xspace}                 
 \def\PW      {\ensuremath{\mathrm{W}}\xspace}                 
 \def\PX      {\ensuremath{\mathrm{X}}\xspace}                 
 \def\PY      {\ensuremath{\mathrm{Y}}\xspace}                 
 \def\PZ      {\ensuremath{\mathrm{Z}}\xspace}                 
 \def\Pa      {\ensuremath{\mathrm{a}}\xspace}                 
 \def\Pb      {\ensuremath{\mathrm{b}}\xspace}                 
 \def\Pc      {\ensuremath{\mathrm{c}}\xspace}                 
 \def\Pd      {\ensuremath{\mathrm{d}}\xspace}                 
 \def\Pe      {\ensuremath{\mathrm{e}}\xspace}                 
 \def\Pf      {\ensuremath{\mathrm{f}}\xspace}                 
 \def\Pg      {\ensuremath{\mathrm{g}}\xspace}                 
 \def\Ph      {\ensuremath{\mathrm{h}}\xspace}                 
% \def\Pi      {\ensuremath{\mathrm{i}}\xspace}                 
 \def\Pj      {\ensuremath{\mathrm{j}}\xspace}                 
 \def\Pk      {\ensuremath{\mathrm{k}}\xspace}                 
 \def\Pl      {\ensuremath{\mathrm{l}}\xspace}                 
 \def\Pm      {\ensuremath{\mathrm{m}}\xspace}                 
 \def\Pn      {\ensuremath{\mathrm{n}}\xspace}                 
 \def\Po      {\ensuremath{\mathrm{o}}\xspace}                 
 \def\Pp      {\ensuremath{\mathrm{p}}\xspace}                 
 \def\Pq      {\ensuremath{\mathrm{q}}\xspace}                 
 \def\Pr      {\ensuremath{\mathrm{r}}\xspace}                 
 \def\Ps      {\ensuremath{\mathrm{s}}\xspace}                 
 \def\Pt      {\ensuremath{\mathrm{t}}\xspace}                 
 \def\Pu      {\ensuremath{\mathrm{u}}\xspace}                 
 \def\Pv      {\ensuremath{\mathrm{v}}\xspace}                 
 \def\Pw      {\ensuremath{\mathrm{w}}\xspace}                 
 \def\Px      {\ensuremath{\mathrm{x}}\xspace}                 
 \def\Py      {\ensuremath{\mathrm{y}}\xspace}                 
 \def\Pz      {\ensuremath{\mathrm{z}}\xspace}                 
}
{\def\Palpha      {\ensuremath{\alpha}\xspace}
 \def\Pbeta       {\ensuremath{\beta}\xspace}
 \def\Pgamma      {\ensuremath{\gamma}\xspace}                 
 \def\Pdelta      {\ensuremath{\delta}\xspace}                 
 \def\Pepsilon    {\ensuremath{\epsilon}\xspace}                 
 \def\Pvarepsilon {\ensuremath{\varepsilon}\xspace}                 
 \def\Pzeta       {\ensuremath{\zeta}\xspace}                 
 \def\Peta        {\ensuremath{\eta}\xspace}                 
 \def\Ptheta      {\ensuremath{\theta}\xspace}                 
 \def\Pvartheta   {\ensuremath{\vartheta}\xspace}                 
 \def\Piota       {\ensuremath{\iota}\xspace}                 
 \def\Pkappa      {\ensuremath{\kappa}\xspace}                 
 \def\Plambda     {\ensuremath{\lambda}\xspace}                 
 \def\Pmu         {\ensuremath{\mu}\xspace}                 
 \def\Pnu         {\ensuremath{\nu}\xspace}                 
 \def\Pxi         {\ensuremath{\xi}\xspace}                 
 \def\Ppi         {\ensuremath{\pi}\xspace}                 
 \def\Pvarpi      {\ensuremath{\varpi}\xspace}                 
 \def\Prho        {\ensuremath{\rho}\xspace}                 
 \def\Pvarrho     {\ensuremath{\varrho}\xspace}                 
 \def\Ptau        {\ensuremath{\tau}\xspace}                 
 \def\Pupsilon    {\ensuremath{\upsilon}\xspace}                 
 \def\Pphi        {\ensuremath{\phi}\xspace}                 
 \def\Pvarphi     {\ensuremath{\varphi}\xspace}                 
 \def\Pchi        {\ensuremath{\chi}\xspace}                 
 \def\Ppsi        {\ensuremath{\psi}\xspace}                 
 \def\Pomega      {\ensuremath{\omega}\xspace}                 
 \mathchardef\PDelta="7101
 \mathchardef\PXi="7104
 \mathchardef\PLambda="7103
 \mathchardef\PSigma="7106
 \mathchardef\POmega="710A
 \mathchardef\PUpsilon="7107
 \def\PA      {\ensuremath{A}\xspace}                 
 \def\PB      {\ensuremath{B}\xspace}                 
 \def\PC      {\ensuremath{C}\xspace}                 
 \def\PD      {\ensuremath{D}\xspace}                 
 \def\PE      {\ensuremath{E}\xspace}                 
 \def\PF      {\ensuremath{F}\xspace}                 
 \def\PG      {\ensuremath{G}\xspace}                 
 \def\PH      {\ensuremath{H}\xspace}                 
 \def\PI      {\ensuremath{I}\xspace}                 
 \def\PJ      {\ensuremath{J}\xspace}                 
 \def\PK      {\ensuremath{K}\xspace}                 
 \def\PL      {\ensuremath{L}\xspace}                 
 \def\PM      {\ensuremath{M}\xspace}                 
 \def\PN      {\ensuremath{N}\xspace}                 
 \def\PO      {\ensuremath{O}\xspace}                 
 \def\PP      {\ensuremath{P}\xspace}                 
 \def\PQ      {\ensuremath{Q}\xspace}                 
 \def\PR      {\ensuremath{R}\xspace}                 
 \def\PS      {\ensuremath{S}\xspace}                 
 \def\PT      {\ensuremath{T}\xspace}                 
 \def\PU      {\ensuremath{U}\xspace}                 
 \def\PV      {\ensuremath{V}\xspace}                 
 \def\PW      {\ensuremath{W}\xspace}                 
 \def\PX      {\ensuremath{X}\xspace}                 
 \def\PY      {\ensuremath{Y}\xspace}                 
 \def\PZ      {\ensuremath{Z}\xspace}                 
 \def\Pa      {\ensuremath{a}\xspace}                 
 \def\Pb      {\ensuremath{b}\xspace}                 
 \def\Pc      {\ensuremath{c}\xspace}                 
 \def\Pd      {\ensuremath{d}\xspace}                 
 \def\Pe      {\ensuremath{e}\xspace}                 
 \def\Pf      {\ensuremath{f}\xspace}                 
 \def\Pg      {\ensuremath{g}\xspace}                 
 \def\Ph      {\ensuremath{h}\xspace}                 
% \def\Pi      {\ensuremath{i}\xspace}                 
 \def\Pj      {\ensuremath{j}\xspace}                 
 \def\Pk      {\ensuremath{k}\xspace}                 
 \def\Pl      {\ensuremath{l}\xspace}                 
 \def\Pm      {\ensuremath{m}\xspace}                 
 \def\Pn      {\ensuremath{n}\xspace}                 
 \def\Po      {\ensuremath{o}\xspace}                 
 \def\Pp      {\ensuremath{p}\xspace}                 
 \def\Pq      {\ensuremath{q}\xspace}                 
 \def\Pr      {\ensuremath{r}\xspace}                 
 \def\Ps      {\ensuremath{s}\xspace}                 
 \def\Pt      {\ensuremath{t}\xspace}                 
 \def\Pu      {\ensuremath{u}\xspace}                 
 \def\Pv      {\ensuremath{v}\xspace}                 
 \def\Pw      {\ensuremath{w}\xspace}                 
 \def\Px      {\ensuremath{x}\xspace}                 
 \def\Py      {\ensuremath{y}\xspace}                 
 \def\Pz      {\ensuremath{z}\xspace}                 
}

%%%%%%%%%%%%%%%%%%%%%%%%%%%%%%%%%%%%%%%%%%%%%%%
% Particles
\makeatletter
\ifcase \@ptsize \relax% 10pt
  \newcommand{\miniscule}{\@setfontsize\miniscule{4}{5}}% \tiny: 5/6
\or% 11pt
  \newcommand{\miniscule}{\@setfontsize\miniscule{5}{6}}% \tiny: 6/7
\or% 12pt
  \newcommand{\miniscule}{\@setfontsize\miniscule{5}{6}}% \tiny: 6/7
\fi
\makeatother


\DeclareRobustCommand{\optbar}[1]{\shortstack{{\miniscule (\rule[.5ex]{1.25em}{.18mm})}
  \\ [-.7ex] $#1$}}


%% Leptons

\let\emi\en
\def\electron   {{\ensuremath{\Pe}}\xspace}
\def\en         {{\ensuremath{\Pe^-}}\xspace}   % electron negative (\em is taken)
\def\ep         {{\ensuremath{\Pe^+}}\xspace}
\def\epm        {{\ensuremath{\Pe^\pm}}\xspace} 
\def\epem       {{\ensuremath{\Pe^+\Pe^-}}\xspace}
%%%\def\ee         {\ensuremath{\Pe^-\Pe^-}\xspace}

\def\muon       {{\ensuremath{\Pmu}}\xspace}
\def\mup        {{\ensuremath{\Pmu^+}}\xspace}
\def\mun        {{\ensuremath{\Pmu^-}}\xspace} % muon negative (\mum is taken)
\def\mumu       {{\ensuremath{\Pmu^+\Pmu^-}}\xspace}

\def\tauon      {{\ensuremath{\Ptau}}\xspace}
\def\taup       {{\ensuremath{\Ptau^+}}\xspace}
\def\taum       {{\ensuremath{\Ptau^-}}\xspace}
\def\tautau     {{\ensuremath{\Ptau^+\Ptau^-}}\xspace}

\def\lepton     {{\ensuremath{\ell}}\xspace}
\def\ellm       {{\ensuremath{\ell^-}}\xspace}
\def\ellp       {{\ensuremath{\ell^+}}\xspace}
%%%\def\ellell     {\ensuremath{\ell^+ \ell^-}\xspace}

\def\neu        {{\ensuremath{\Pnu}}\xspace}
\def\neub       {{\ensuremath{\overline{\Pnu}}}\xspace}
%%%\def\nuenueb    {\ensuremath{\neu\neub}\xspace}
\def\neue       {{\ensuremath{\neu_e}}\xspace}
\def\neueb      {{\ensuremath{\neub_e}}\xspace}
%%%\def\neueneueb  {\ensuremath{\neue\neueb}\xspace}
\def\neum       {{\ensuremath{\neu_\mu}}\xspace}
\def\neumb      {{\ensuremath{\neub_\mu}}\xspace}
%%%\def\neumneumb  {\ensuremath{\neum\neumb}\xspace}
\def\neut       {{\ensuremath{\neu_\tau}}\xspace}
\def\neutb      {{\ensuremath{\neub_\tau}}\xspace}
%%%\def\neutneutb  {\ensuremath{\neut\neutb}\xspace}
\def\neul       {{\ensuremath{\neu_\ell}}\xspace}
\def\neulb      {{\ensuremath{\neub_\ell}}\xspace}
%%%\def\neulneulb  {\ensuremath{\neul\neulb}\xspace}

%% Gauge bosons and scalars

\def\g      {{\ensuremath{\Pgamma}}\xspace}
\def\H      {{\ensuremath{\PH^0}}\xspace}
\def\Hp     {{\ensuremath{\PH^+}}\xspace}
\def\Hm     {{\ensuremath{\PH^-}}\xspace}
\def\Hpm    {{\ensuremath{\PH^\pm}}\xspace}
\def\W      {{\ensuremath{\PW}}\xspace}
\def\Wp     {{\ensuremath{\PW^+}}\xspace}
\def\Wm     {{\ensuremath{\PW^-}}\xspace}
\def\Wpm    {{\ensuremath{\PW^\pm}}\xspace}
\def\Z      {{\ensuremath{\PZ}}\xspace}

%% Quarks

\def\quark     {{\ensuremath{\Pq}}\xspace}
\def\quarkbar  {{\ensuremath{\overline \quark}}\xspace}
\def\qqbar     {{\ensuremath{\quark\quarkbar}}\xspace}
\def\uquark    {{\ensuremath{\Pu}}\xspace}
\def\uquarkbar {{\ensuremath{\overline \uquark}}\xspace}
\def\uubar     {{\ensuremath{\uquark\uquarkbar}}\xspace}
\def\dquark    {{\ensuremath{\Pd}}\xspace}
\def\dquarkbar {{\ensuremath{\overline \dquark}}\xspace}
\def\ddbar     {{\ensuremath{\dquark\dquarkbar}}\xspace}
\def\squark    {{\ensuremath{\Ps}}\xspace}
\def\squarkbar {{\ensuremath{\overline \squark}}\xspace}
\def\ssbar     {{\ensuremath{\squark\squarkbar}}\xspace}
\def\cquark    {{\ensuremath{\Pc}}\xspace}
\def\cquarkbar {{\ensuremath{\overline \cquark}}\xspace}
\def\ccbar     {{\ensuremath{\cquark\cquarkbar}}\xspace}
\def\bquark    {{\ensuremath{\Pb}}\xspace}
\def\bquarkbar {{\ensuremath{\overline \bquark}}\xspace}
\def\bbbar     {{\ensuremath{\bquark\bquarkbar}}\xspace}
\def\tquark    {{\ensuremath{\Pt}}\xspace}
\def\tquarkbar {{\ensuremath{\overline \tquark}}\xspace}
\def\ttbar     {{\ensuremath{\tquark\tquarkbar}}\xspace}

%% Light mesons

\def\hadron {{\ensuremath{\Ph}}\xspace}
\def\pion   {{\ensuremath{\Ppi}}\xspace}
\def\piz    {{\ensuremath{\pion^0}}\xspace}
\def\pizs   {{\ensuremath{\pion^0\mbox\,\mathrm{s}}}\xspace}
\def\pip    {{\ensuremath{\pion^+}}\xspace}
\def\pim    {{\ensuremath{\pion^-}}\xspace}
\def\pipm   {{\ensuremath{\pion^\pm}}\xspace}
\def\pimp   {{\ensuremath{\pion^\mp}}\xspace}

\def\rhomeson {{\ensuremath{\Prho}}\xspace}
\def\rhoz     {{\ensuremath{\rhomeson^0}}\xspace}
\def\rhop     {{\ensuremath{\rhomeson^+}}\xspace}
\def\rhom     {{\ensuremath{\rhomeson^-}}\xspace}
\def\rhopm    {{\ensuremath{\rhomeson^\pm}}\xspace}
\def\rhomp    {{\ensuremath{\rhomeson^\mp}}\xspace}

\def\kaon    {{\ensuremath{\PK}}\xspace}
%%% do NOT use ensuremath here
  \def\Kbar    {{\kern 0.2em\overline{\kern -0.2em \PK}{}}\xspace}
\def\Kb      {{\ensuremath{\Kbar}}\xspace}
\def\KorKbar    {\kern 0.18em\optbar{\kern -0.18em K}{}\xspace}
\def\Kz      {{\ensuremath{\kaon^0}}\xspace}
\def\Kzb     {{\ensuremath{\Kbar{}^0}}\xspace}
\def\Kp      {{\ensuremath{\kaon^+}}\xspace}
\def\Km      {{\ensuremath{\kaon^-}}\xspace}
\def\Kpm     {{\ensuremath{\kaon^\pm}}\xspace}
\def\Kmp     {{\ensuremath{\kaon^\mp}}\xspace}
\def\KS      {{\ensuremath{\kaon^0_{\mathrm{ \scriptscriptstyle S}}}}\xspace}
\def\KL      {{\ensuremath{\kaon^0_{\mathrm{ \scriptscriptstyle L}}}}\xspace}
\def\Kstarz  {{\ensuremath{\kaon^{*0}}}\xspace}
\def\Kstarzb {{\ensuremath{\Kbar{}^{*0}}}\xspace}
\def\Kstar   {{\ensuremath{\kaon^*}}\xspace}
\def\Kstarb  {{\ensuremath{\Kbar{}^*}}\xspace}
\def\Kstarp  {{\ensuremath{\kaon^{*+}}}\xspace}
\def\Kstarm  {{\ensuremath{\kaon^{*-}}}\xspace}
\def\Kstarpm {{\ensuremath{\kaon^{*\pm}}}\xspace}
\def\Kstarmp {{\ensuremath{\kaon^{*\mp}}}\xspace}

\newcommand{\etaz}{\ensuremath{\Peta}\xspace}
\newcommand{\etapr}{\ensuremath{\Peta^{\prime}}\xspace}
\newcommand{\phiz}{\ensuremath{\Pphi}\xspace}
\newcommand{\omegaz}{\ensuremath{\Pomega}\xspace}

%% Heavy mesons

%%% do NOT use ensuremath here
  \def\Dbar    {{\kern 0.2em\overline{\kern -0.2em \PD}{}}\xspace}
\def\D       {{\ensuremath{\PD}}\xspace}
\def\Db      {{\ensuremath{\Dbar}}\xspace}
\def\DorDbar    {\kern 0.18em\optbar{\kern -0.18em D}{}\xspace}
\def\Dz      {{\ensuremath{\D^0}}\xspace}
\def\Dzb     {{\ensuremath{\Dbar{}^0}}\xspace}
\def\Dp      {{\ensuremath{\D^+}}\xspace}
\def\Dm      {{\ensuremath{\D^-}}\xspace}
\def\Dpm     {{\ensuremath{\D^\pm}}\xspace}
\def\Dmp     {{\ensuremath{\D^\mp}}\xspace}
\def\Dstar   {{\ensuremath{\D^*}}\xspace}
\def\Dstarb  {{\ensuremath{\Dbar{}^*}}\xspace}
\def\Dstarz  {{\ensuremath{\D^{*0}}}\xspace}
\def\Dstarzb {{\ensuremath{\Dbar{}^{*0}}}\xspace}
\def\Dstarp  {{\ensuremath{\D^{*+}}}\xspace}
\def\Dstarm  {{\ensuremath{\D^{*-}}}\xspace}
\def\Dstarpm {{\ensuremath{\D^{*\pm}}}\xspace}
\def\Dstarmp {{\ensuremath{\D^{*\mp}}}\xspace}
\def\Ds      {{\ensuremath{\D^+_\squark}}\xspace}
\def\Dsp     {{\ensuremath{\D^+_\squark}}\xspace}
\def\Dsm     {{\ensuremath{\D^-_\squark}}\xspace}
\def\Dspm    {{\ensuremath{\D^{\pm}_\squark}}\xspace}
\def\Dsmp    {{\ensuremath{\D^{\mp}_\squark}}\xspace}
\def\Dss     {{\ensuremath{\D^{*+}_\squark}}\xspace}
\def\Dssp    {{\ensuremath{\D^{*+}_\squark}}\xspace}
\def\Dssm    {{\ensuremath{\D^{*-}_\squark}}\xspace}
\def\Dsspm   {{\ensuremath{\D^{*\pm}_\squark}}\xspace}
\def\Dssmp   {{\ensuremath{\D^{*\mp}_\squark}}\xspace}

\def\B       {{\ensuremath{\PB}}\xspace}
%%% do NOT use ensuremath here
\def\Bbar    {{\ensuremath{\kern 0.18em\overline{\kern -0.18em \PB}{}}}\xspace}
\def\Bb      {{\ensuremath{\Bbar}}\xspace}
\def\BorBbar    {\kern 0.18em\optbar{\kern -0.18em B}{}\xspace}
\def\Bz      {{\ensuremath{\B^0}}\xspace}
\def\Bzb     {{\ensuremath{\Bbar{}^0}}\xspace}
\def\Bu      {{\ensuremath{\B^+}}\xspace}
\def\Bub     {{\ensuremath{\B^-}}\xspace}
\def\Bp      {{\ensuremath{\Bu}}\xspace}
\def\Bm      {{\ensuremath{\Bub}}\xspace}
\def\Bpm     {{\ensuremath{\B^\pm}}\xspace}
\def\Bmp     {{\ensuremath{\B^\mp}}\xspace}
\def\Bd      {{\ensuremath{\B^0}}\xspace}
\def\Bs      {{\ensuremath{\B^0_\squark}}\xspace}
\def\Bsb     {{\ensuremath{\Bbar{}^0_\squark}}\xspace}
\def\Bdb     {{\ensuremath{\Bbar{}^0}}\xspace}
\def\Bc      {{\ensuremath{\B_\cquark^+}}\xspace}
\def\Bcp     {{\ensuremath{\B_\cquark^+}}\xspace}
\def\Bcm     {{\ensuremath{\B_\cquark^-}}\xspace}
\def\Bcpm    {{\ensuremath{\B_\cquark^\pm}}\xspace}

%% Onia

\def\jpsi     {{\ensuremath{{\PJ\mskip -3mu/\mskip -2mu\Ppsi\mskip 2mu}}}\xspace}
\def\psitwos  {{\ensuremath{\Ppsi{(2S)}}}\xspace}
\def\psiprpr  {{\ensuremath{\Ppsi(3770)}}\xspace}
\def\etac     {{\ensuremath{\Peta_\cquark}}\xspace}
\def\chiczero {{\ensuremath{\Pchi_{\cquark 0}}}\xspace}
\def\chicone  {{\ensuremath{\Pchi_{\cquark 1}}}\xspace}
\def\chictwo  {{\ensuremath{\Pchi_{\cquark 2}}}\xspace}
  %\mathchardef\Upsilon="7107
  \def\Y#1S{\ensuremath{\PUpsilon{(#1S)}}\xspace}% no space before {...}!
\def\OneS  {{\Y1S}}
\def\TwoS  {{\Y2S}}
\def\ThreeS{{\Y3S}}
\def\FourS {{\Y4S}}
\def\FiveS {{\Y5S}}

\def\chic  {{\ensuremath{\Pchi_{c}}}\xspace}

%% Baryons

\def\proton      {{\ensuremath{\Pp}}\xspace}
\def\antiproton  {{\ensuremath{\overline \proton}}\xspace}
\def\neutron     {{\ensuremath{\Pn}}\xspace}
\def\antineutron {{\ensuremath{\overline \neutron}}\xspace}
\def\Deltares    {{\ensuremath{\PDelta}}\xspace}
\def\Deltaresbar {{\ensuremath{\overline \Deltares}}\xspace}
\def\Xires       {{\ensuremath{\PXi}}\xspace}
\def\Xiresbar    {{\ensuremath{\overline \Xires}}\xspace}
\def\Lz          {{\ensuremath{\PLambda}}\xspace}
\def\Lbar        {{\ensuremath{\kern 0.1em\overline{\kern -0.1em\PLambda}}}\xspace}
\def\LorLbar    {\kern 0.18em\optbar{\kern -0.18em \PLambda}{}\xspace}
\def\Lambdares   {{\ensuremath{\PLambda}}\xspace}
\def\Lambdaresbar{{\ensuremath{\Lbar}}\xspace}
\def\Sigmares    {{\ensuremath{\PSigma}}\xspace}
\def\Sigmaresbar {{\ensuremath{\overline \Sigmares}}\xspace}
\def\Omegares    {{\ensuremath{\POmega}}\xspace}
\def\Omegaresbar {{\ensuremath{\overline \POmega}}\xspace}

%%% do NOT use ensuremath here
 % \def\Deltabar{\kern 0.25em\overline{\kern -0.25em \Deltares}{}\xspace}
 % \def\Sigbar{\kern 0.2em\overline{\kern -0.2em \Sigma}{}\xspace}
 % \def\Xibar{\kern 0.2em\overline{\kern -0.2em \Xi}{}\xspace}
 % \def\Obar{\kern 0.2em\overline{\kern -0.2em \Omega}{}\xspace}
 % \def\Nbar{\kern 0.2em\overline{\kern -0.2em N}{}\xspace}
 % \def\Xb{\kern 0.2em\overline{\kern -0.2em X}{}\xspace}

\def\Lb      {{\ensuremath{\Lz^0_\bquark}}\xspace}
\def\Lbbar   {{\ensuremath{\Lbar{}^0_\bquark}}\xspace}
\def\Lc      {{\ensuremath{\Lz^+_\cquark}}\xspace}
\def\Lcbar   {{\ensuremath{\Lbar{}^-_\cquark}}\xspace}
\def\Xib     {{\ensuremath{\Xires_\bquark}}\xspace}
\def\Xibz    {{\ensuremath{\Xires^0_\bquark}}\xspace}
\def\Xibm    {{\ensuremath{\Xires^-_\bquark}}\xspace}
\def\Xibbar  {{\ensuremath{\Xiresbar{}_\bquark}}\xspace}
\def\Xibbarz {{\ensuremath{\Xiresbar{}_\bquark^0}}\xspace}
\def\Xibbarp {{\ensuremath{\Xiresbar{}_\bquark^+}}\xspace}
\def\Xic     {{\ensuremath{\Xires_\cquark}}\xspace}
\def\Xicz    {{\ensuremath{\Xires^0_\cquark}}\xspace}
\def\Xicp    {{\ensuremath{\Xires^+_\cquark}}\xspace}
\def\Xicbar  {{\ensuremath{\Xiresbar{}_\cquark}}\xspace}
\def\Xicbarz {{\ensuremath{\Xiresbar{}_\cquark^0}}\xspace}
\def\Xicbarm {{\ensuremath{\Xiresbar{}_\cquark^-}}\xspace}
\def\Omegac    {{\ensuremath{\Omegares^0_\cquark}}\xspace}
\def\Omegacbar {{\ensuremath{\Omegaresbar{}_\cquark^0}}\xspace}
\def\Omegab    {{\ensuremath{\Omegares^-_\bquark}}\xspace}
\def\Omegabbar {{\ensuremath{\Omegaresbar{}_\bquark^+}}\xspace}

%%%%%%%%%%%%%%%%%%
% Physics symbols
%%%%%%%%%%%%%%%%%

%% Decays
\def\BF         {{\ensuremath{\mathcal{B}}}\xspace}
\def\BRvis      {{\ensuremath{\BR_{\mathrm{{vis}}}}}}
\def\BR         {\BF}
\newcommand{\decay}[2]{\ensuremath{#1\!\to #2}\xspace}         % {\Pa}{\Pb \Pc}
\def\ra                 {\ensuremath{\rightarrow}\xspace}
\def\to                 {\ensuremath{\rightarrow}\xspace}

%% Lifetimes
\newcommand{\tauBs}{{\ensuremath{\tau_{\Bs}}}\xspace}
\newcommand{\tauBd}{{\ensuremath{\tau_{\Bd}}}\xspace}
\newcommand{\tauBz}{{\ensuremath{\tau_{\Bz}}}\xspace}
\newcommand{\tauBu}{{\ensuremath{\tau_{\Bp}}}\xspace}
\newcommand{\tauDp}{{\ensuremath{\tau_{\Dp}}}\xspace}
\newcommand{\tauDz}{{\ensuremath{\tau_{\Dz}}}\xspace}
\newcommand{\tauL}{{\ensuremath{\tau_{\mathrm{ L}}}}\xspace}
\newcommand{\tauH}{{\ensuremath{\tau_{\mathrm{ H}}}}\xspace}

%% Masses
\newcommand{\mBd}{{\ensuremath{m_{\Bd}}}\xspace}
\newcommand{\mBp}{{\ensuremath{m_{\Bp}}}\xspace}
\newcommand{\mBs}{{\ensuremath{m_{\Bs}}}\xspace}
\newcommand{\mBc}{{\ensuremath{m_{\Bc}}}\xspace}
\newcommand{\mLb}{{\ensuremath{m_{\Lb}}}\xspace}

%% EW theory, groups
\def\grpsuthree {{\ensuremath{\mathrm{SU}(3)}}\xspace}
\def\grpsutw    {{\ensuremath{\mathrm{SU}(2)}}\xspace}
\def\grpuone    {{\ensuremath{\mathrm{U}(1)}}\xspace}

\def\ssqtw   {{\ensuremath{\sin^{2}\!\theta_{\mathrm{W}}}}\xspace}
\def\csqtw   {{\ensuremath{\cos^{2}\!\theta_{\mathrm{W}}}}\xspace}
\def\stw     {{\ensuremath{\sin\theta_{\mathrm{W}}}}\xspace}
\def\ctw     {{\ensuremath{\cos\theta_{\mathrm{W}}}}\xspace}
\def\ssqtwef {{\ensuremath{{\sin}^{2}\theta_{\mathrm{W}}^{\mathrm{eff}}}}\xspace}
\def\csqtwef {{\ensuremath{{\cos}^{2}\theta_{\mathrm{W}}^{\mathrm{eff}}}}\xspace}
\def\stwef   {{\ensuremath{\sin\theta_{\mathrm{W}}^{\mathrm{eff}}}}\xspace}
\def\ctwef   {{\ensuremath{\cos\theta_{\mathrm{W}}^{\mathrm{eff}}}}\xspace}
\def\gv      {{\ensuremath{g_{\mbox{\tiny V}}}}\xspace}
\def\ga      {{\ensuremath{g_{\mbox{\tiny A}}}}\xspace}

\def\order   {{\ensuremath{\mathcal{O}}}\xspace}
\def\ordalph {{\ensuremath{\mathcal{O}(\alpha)}}\xspace}
\def\ordalsq {{\ensuremath{\mathcal{O}(\alpha^{2})}}\xspace}
\def\ordalcb {{\ensuremath{\mathcal{O}(\alpha^{3})}}\xspace}

%% QCD parameters
\newcommand{\as}{{\ensuremath{\alpha_s}}\xspace}
\newcommand{\MSb}{{\ensuremath{\overline{\mathrm{MS}}}}\xspace}
\newcommand{\lqcd}{{\ensuremath{\Lambda_{\mathrm{QCD}}}}\xspace}
\def\qsq       {{\ensuremath{q^2}}\xspace}

%% CKM, CP violation

\def\eps   {{\ensuremath{\varepsilon}}\xspace}
\def\epsK  {{\ensuremath{\varepsilon_K}}\xspace}
\def\epsB  {{\ensuremath{\varepsilon_B}}\xspace}
\def\epsp  {{\ensuremath{\varepsilon^\prime_K}}\xspace}

\def\CP                {{\ensuremath{C\!P}}\xspace}
\def\CPT               {{\ensuremath{C\!PT}}\xspace}

\def\rhobar {{\ensuremath{\overline \rho}}\xspace}
\def\etabar {{\ensuremath{\overline \eta}}\xspace}

\def\Vud  {{\ensuremath{V_{\uquark\dquark}}}\xspace}
\def\Vcd  {{\ensuremath{V_{\cquark\dquark}}}\xspace}
\def\Vtd  {{\ensuremath{V_{\tquark\dquark}}}\xspace}
\def\Vus  {{\ensuremath{V_{\uquark\squark}}}\xspace}
\def\Vcs  {{\ensuremath{V_{\cquark\squark}}}\xspace}
\def\Vts  {{\ensuremath{V_{\tquark\squark}}}\xspace}
\def\Vub  {{\ensuremath{V_{\uquark\bquark}}}\xspace}
\def\Vcb  {{\ensuremath{V_{\cquark\bquark}}}\xspace}
\def\Vtb  {{\ensuremath{V_{\tquark\bquark}}}\xspace}
\def\Vuds  {{\ensuremath{V_{\uquark\dquark}^\ast}}\xspace}
\def\Vcds  {{\ensuremath{V_{\cquark\dquark}^\ast}}\xspace}
\def\Vtds  {{\ensuremath{V_{\tquark\dquark}^\ast}}\xspace}
\def\Vuss  {{\ensuremath{V_{\uquark\squark}^\ast}}\xspace}
\def\Vcss  {{\ensuremath{V_{\cquark\squark}^\ast}}\xspace}
\def\Vtss  {{\ensuremath{V_{\tquark\squark}^\ast}}\xspace}
\def\Vubs  {{\ensuremath{V_{\uquark\bquark}^\ast}}\xspace}
\def\Vcbs  {{\ensuremath{V_{\cquark\bquark}^\ast}}\xspace}
\def\Vtbs  {{\ensuremath{V_{\tquark\bquark}^\ast}}\xspace}

%% Oscillations

\newcommand{\dm}{{\ensuremath{\Delta m}}\xspace}
\newcommand{\dms}{{\ensuremath{\Delta m_{\squark}}}\xspace}
\newcommand{\dmd}{{\ensuremath{\Delta m_{\dquark}}}\xspace}
\newcommand{\DG}{{\ensuremath{\Delta\Gamma}}\xspace}
\newcommand{\DGs}{{\ensuremath{\Delta\Gamma_{\squark}}}\xspace}
\newcommand{\DGd}{{\ensuremath{\Delta\Gamma_{\dquark}}}\xspace}
\newcommand{\Gs}{{\ensuremath{\Gamma_{\squark}}}\xspace}
\newcommand{\Gd}{{\ensuremath{\Gamma_{\dquark}}}\xspace}
\newcommand{\MBq}{{\ensuremath{M_{\B_\quark}}}\xspace}
\newcommand{\DGq}{{\ensuremath{\Delta\Gamma_{\quark}}}\xspace}
\newcommand{\Gq}{{\ensuremath{\Gamma_{\quark}}}\xspace}
\newcommand{\dmq}{{\ensuremath{\Delta m_{\quark}}}\xspace}
\newcommand{\GL}{{\ensuremath{\Gamma_{\mathrm{ L}}}}\xspace}
\newcommand{\GH}{{\ensuremath{\Gamma_{\mathrm{ H}}}}\xspace}
\newcommand{\DGsGs}{{\ensuremath{\Delta\Gamma_{\squark}/\Gamma_{\squark}}}\xspace}
\newcommand{\Delm}{{\mbox{$\Delta m $}}\xspace}
\newcommand{\ACP}{{\ensuremath{{\mathcal{A}}^{\CP}}}\xspace}
\newcommand{\Adir}{{\ensuremath{{\mathcal{A}}^{\mathrm{ dir}}}}\xspace}
\newcommand{\Amix}{{\ensuremath{{\mathcal{A}}^{\mathrm{ mix}}}}\xspace}
\newcommand{\ADelta}{{\ensuremath{{\mathcal{A}}^\Delta}}\xspace}
\newcommand{\phid}{{\ensuremath{\phi_{\dquark}}}\xspace}
\newcommand{\sinphid}{{\ensuremath{\sin\!\phid}}\xspace}
\newcommand{\phis}{{\ensuremath{\phi_{\squark}}}\xspace}
\newcommand{\betas}{{\ensuremath{\beta_{\squark}}}\xspace}
\newcommand{\sbetas}{{\ensuremath{\sigma(\beta_{\squark})}}\xspace}
\newcommand{\stbetas}{{\ensuremath{\sigma(2\beta_{\squark})}}\xspace}
\newcommand{\stphis}{{\ensuremath{\sigma(\phi_{\squark})}}\xspace}
\newcommand{\sinphis}{{\ensuremath{\sin\!\phis}}\xspace}

%% Tagging
\newcommand{\edet}{{\ensuremath{\varepsilon_{\mathrm{ det}}}}\xspace}
\newcommand{\erec}{{\ensuremath{\varepsilon_{\mathrm{ rec/det}}}}\xspace}
\newcommand{\esel}{{\ensuremath{\varepsilon_{\mathrm{ sel/rec}}}}\xspace}
\newcommand{\etrg}{{\ensuremath{\varepsilon_{\mathrm{ trg/sel}}}}\xspace}
\newcommand{\etot}{{\ensuremath{\varepsilon_{\mathrm{ tot}}}}\xspace}

\newcommand{\mistag}{\ensuremath{\omega}\xspace}
\newcommand{\wcomb}{\ensuremath{\omega^{\mathrm{comb}}}\xspace}
\newcommand{\etag}{{\ensuremath{\varepsilon_{\mathrm{tag}}}}\xspace}
\newcommand{\etagcomb}{{\ensuremath{\varepsilon_{\mathrm{tag}}^{\mathrm{comb}}}}\xspace}
\newcommand{\effeff}{\ensuremath{\varepsilon_{\mathrm{eff}}}\xspace}
\newcommand{\effeffcomb}{\ensuremath{\varepsilon_{\mathrm{eff}}^{\mathrm{comb}}}\xspace}
\newcommand{\efftag}{{\ensuremath{\etag(1-2\omega)^2}}\xspace}
\newcommand{\effD}{{\ensuremath{\etag D^2}}\xspace}

\newcommand{\etagprompt}{{\ensuremath{\varepsilon_{\mathrm{ tag}}^{\mathrm{Pr}}}}\xspace}
\newcommand{\etagLL}{{\ensuremath{\varepsilon_{\mathrm{ tag}}^{\mathrm{LL}}}}\xspace}

%% Key decay channels

\def\BdToKstmm    {\decay{\Bd}{\Kstarz\mup\mun}}
\def\BdbToKstmm   {\decay{\Bdb}{\Kstarzb\mup\mun}}

\def\BsToJPsiPhi  {\decay{\Bs}{\jpsi\phi}}
\def\BdToJPsiKst  {\decay{\Bd}{\jpsi\Kstarz}}
\def\BdbToJPsiKst {\decay{\Bdb}{\jpsi\Kstarzb}}

\def\BsPhiGam     {\decay{\Bs}{\phi \g}}
\def\BdKstGam     {\decay{\Bd}{\Kstarz \g}}

\def\BTohh        {\decay{\B}{\Ph^+ \Ph'^-}}
\def\BdTopipi     {\decay{\Bd}{\pip\pim}}
\def\BdToKpi      {\decay{\Bd}{\Kp\pim}}
\def\BsToKK       {\decay{\Bs}{\Kp\Km}}
\def\BsTopiK      {\decay{\Bs}{\pip\Km}}

%% Rare decays
\def\BdKstee  {\decay{\Bd}{\Kstarz\epem}}
\def\BdbKstee {\decay{\Bdb}{\Kstarzb\epem}}
\def\bsll     {\decay{\bquark}{\squark \ell^+ \ell^-}}
\def\AFB      {\ensuremath{A_{\mathrm{FB}}}\xspace}
\def\FL       {\ensuremath{F_{\mathrm{L}}}\xspace}
\def\AT#1     {\ensuremath{A_{\mathrm{T}}^{#1}}\xspace}           % 2
\def\btosgam  {\decay{\bquark}{\squark \g}}
\def\btodgam  {\decay{\bquark}{\dquark \g}}
\def\Bsmm     {\decay{\Bs}{\mup\mun}}
\def\Bdmm     {\decay{\Bd}{\mup\mun}}
\def\ctl       {\ensuremath{\cos{\theta_\ell}}\xspace}
\def\ctk       {\ensuremath{\cos{\theta_K}}\xspace}

%% Wilson coefficients and operators
\def\C#1      {\ensuremath{\mathcal{C}_{#1}}\xspace}                       % 9
\def\Cp#1     {\ensuremath{\mathcal{C}_{#1}^{'}}\xspace}                    % 7
\def\Ceff#1   {\ensuremath{\mathcal{C}_{#1}^{\mathrm{(eff)}}}\xspace}        % 9  
\def\Cpeff#1  {\ensuremath{\mathcal{C}_{#1}^{'\mathrm{(eff)}}}\xspace}       % 7
\def\Ope#1    {\ensuremath{\mathcal{O}_{#1}}\xspace}                       % 2
\def\Opep#1   {\ensuremath{\mathcal{O}_{#1}^{'}}\xspace}                    % 7

%% Charm

\def\xprime     {\ensuremath{x^{\prime}}\xspace}
\def\yprime     {\ensuremath{y^{\prime}}\xspace}
\def\ycp        {\ensuremath{y_{\CP}}\xspace}
\def\agamma     {\ensuremath{A_{\Gamma}}\xspace}
%%%\def\kpi        {\ensuremath{\PK\Ppi}\xspace}
%%%\def\kk         {\ensuremath{\PK\PK}\xspace}
%%%\def\dkpi       {\decay{\PD}{\PK\Ppi}}
%%%\def\dkk        {\decay{\PD}{\PK\PK}}
\def\dkpicf     {\decay{\Dz}{\Km\pip}}

%% QM
\newcommand{\bra}[1]{\ensuremath{\langle #1|}}             % {a}
\newcommand{\ket}[1]{\ensuremath{|#1\rangle}}              % {b}
\newcommand{\braket}[2]{\ensuremath{\langle #1|#2\rangle}} % {a}{b}

%%%%%%%%%%%%%%%%%%%%%%%%%%%%%%%%%%%%%%%%%%%%%%%%%%
% Units
%%%%%%%%%%%%%%%%%%%%%%%%%%%%%%%%%%%%%%%%%%%%%%%%%%
\newcommand{\unit}[1]{\ensuremath{\mathrm{ \,#1}}\xspace}          % {kg}

%% Energy and momentum
\newcommand{\tev}{\ifthenelse{\boolean{inbibliography}}{\ensuremath{~T\kern -0.05em eV}\xspace}{\ensuremath{\mathrm{\,Te\kern -0.1em V}}}\xspace}
\newcommand{\gev}{\ensuremath{\mathrm{\,Ge\kern -0.1em V}}\xspace}
\newcommand{\mev}{\ensuremath{\mathrm{\,Me\kern -0.1em V}}\xspace}
\newcommand{\kev}{\ensuremath{\mathrm{\,ke\kern -0.1em V}}\xspace}
\newcommand{\ev}{\ensuremath{\mathrm{\,e\kern -0.1em V}}\xspace}
\newcommand{\gevc}{\ensuremath{{\mathrm{\,Ge\kern -0.1em V\!/}c}}\xspace}
\newcommand{\mevc}{\ensuremath{{\mathrm{\,Me\kern -0.1em V\!/}c}}\xspace}
\newcommand{\gevcc}{\ensuremath{{\mathrm{\,Ge\kern -0.1em V\!/}c^2}}\xspace}
\newcommand{\gevgevcccc}{\ensuremath{{\mathrm{\,Ge\kern -0.1em V^2\!/}c^4}}\xspace}
\newcommand{\mevcc}{\ensuremath{{\mathrm{\,Me\kern -0.1em V\!/}c^2}}\xspace}

%% Distance and area
\def\km   {\ensuremath{\mathrm{ \,km}}\xspace}
\def\m    {\ensuremath{\mathrm{ \,m}}\xspace}
\def\ma   {\ensuremath{{\mathrm{ \,m}}^2}\xspace}
\def\cm   {\ensuremath{\mathrm{ \,cm}}\xspace}
\def\cma  {\ensuremath{{\mathrm{ \,cm}}^2}\xspace}
\def\mm   {\ensuremath{\mathrm{ \,mm}}\xspace}
\def\mma  {\ensuremath{{\mathrm{ \,mm}}^2}\xspace}
\def\mum  {\ensuremath{{\,\upmu\mathrm{m}}}\xspace}
\def\muma {\ensuremath{{\,\upmu\mathrm{m}^2}}\xspace}
\def\nm   {\ensuremath{\mathrm{ \,nm}}\xspace}
\def\fm   {\ensuremath{\mathrm{ \,fm}}\xspace}
\def\barn{\ensuremath{\mathrm{ \,b}}\xspace}
%%%\def\barnhyph{\ensuremath{\mathrm{ -b}}\xspace}
\def\mbarn{\ensuremath{\mathrm{ \,mb}}\xspace}
\def\mub{\ensuremath{{\mathrm{ \,\upmu b}}}\xspace}
%%%\def\mbarnhyph{\ensuremath{\mathrm{ -mb}}\xspace}
\def\nb {\ensuremath{\mathrm{ \,nb}}\xspace}
\def\invnb {\ensuremath{\mbox{\,nb}^{-1}}\xspace}
\def\pb {\ensuremath{\mathrm{ \,pb}}\xspace}
\def\invpb {\ensuremath{\mbox{\,pb}^{-1}}\xspace}
\def\fb   {\ensuremath{\mbox{\,fb}}\xspace}
\def\invfb   {\ensuremath{\mbox{\,fb}^{-1}}\xspace}

%% Time 
\def\sec  {\ensuremath{\mathrm{{\,s}}}\xspace}
\def\ms   {\ensuremath{{\mathrm{ \,ms}}}\xspace}
\def\mus  {\ensuremath{{\,\upmu{\mathrm{ s}}}}\xspace}
\def\ns   {\ensuremath{{\mathrm{ \,ns}}}\xspace}
\def\ps   {\ensuremath{{\mathrm{ \,ps}}}\xspace}
\def\fs   {\ensuremath{\mathrm{ \,fs}}\xspace}

\def\mhz  {\ensuremath{{\mathrm{ \,MHz}}}\xspace}
\def\khz  {\ensuremath{{\mathrm{ \,kHz}}}\xspace}
\def\hz   {\ensuremath{{\mathrm{ \,Hz}}}\xspace}

\def\invps{\ensuremath{{\mathrm{ \,ps^{-1}}}}\xspace}
\def\invns{\ensuremath{{\mathrm{ \,ns^{-1}}}}\xspace}

\def\yr   {\ensuremath{\mathrm{ \,yr}}\xspace}
\def\hr   {\ensuremath{\mathrm{ \,hr}}\xspace}

%% Temperature
\def\degc {\ensuremath{^\circ}{C}\xspace}
\def\degk {\ensuremath {\mathrm{ K}}\xspace}

%% Material lengths, radiation
\def\Xrad {\ensuremath{X_0}\xspace}
\def\NIL{\ensuremath{\lambda_{int}}\xspace}
\def\mip {MIP\xspace}
\def\neutroneq {\ensuremath{\mathrm{ \,n_{eq}}}\xspace}
\def\neqcmcm {\ensuremath{\mathrm{ \,n_{eq} / cm^2}}\xspace}
\def\kRad {\ensuremath{\mathrm{ \,kRad}}\xspace}
\def\MRad {\ensuremath{\mathrm{ \,MRad}}\xspace}
\def\ci {\ensuremath{\mathrm{ \,Ci}}\xspace}
\def\mci {\ensuremath{\mathrm{ \,mCi}}\xspace}

%% Uncertainties
\def\sx    {\ensuremath{\sigma_x}\xspace}    
\def\sy    {\ensuremath{\sigma_y}\xspace}   
\def\sz    {\ensuremath{\sigma_z}\xspace}    

\newcommand{\stat}{\ensuremath{\mathrm{\,(stat)}}\xspace}
\newcommand{\syst}{\ensuremath{\mathrm{\,(syst)}}\xspace}

%% Maths

\def\order{{\ensuremath{\mathcal{O}}}\xspace}
\newcommand{\chisq}{\ensuremath{\chi^2}\xspace}
\newcommand{\chisqndf}{\ensuremath{\chi^2/\mathrm{ndf}}\xspace}
\newcommand{\chisqip}{\ensuremath{\chi^2_{\text{IP}}}\xspace}
\newcommand{\chisqvs}{\ensuremath{\chi^2_{\text{VS}}}\xspace}
\newcommand{\chisqvtx}{\ensuremath{\chi^2_{\text{vtx}}}\xspace}
\newcommand{\chisqvtxndf}{\ensuremath{\chi^2_{\text{vtx}}/\mathrm{ndf}}\xspace}

\def\deriv {\ensuremath{\mathrm{d}}}

\def\gsim{{~\raise.15em\hbox{$>$}\kern-.85em
          \lower.35em\hbox{$\sim$}~}\xspace}
\def\lsim{{~\raise.15em\hbox{$<$}\kern-.85em
          \lower.35em\hbox{$\sim$}~}\xspace}

\newcommand{\mean}[1]{\ensuremath{\left\langle #1 \right\rangle}} % {x}
\newcommand{\abs}[1]{\ensuremath{\left\|#1\right\|}} % {x}
\newcommand{\Real}{\ensuremath{\mathcal{R}e}\xspace}
\newcommand{\Imag}{\ensuremath{\mathcal{I}m}\xspace}

\def\PDF {PDF\xspace}

\def\sPlot{\mbox{\em sPlot}\xspace}
%%%\def\sWeight{\mbox{\em sWeight}\xspace}

%%%%%%%%%%%%%%%%%%%%%%%%%%%%%%%%%%%%%%%%%%%%%%%%%%
% Kinematics
%%%%%%%%%%%%%%%%%%%%%%%%%%%%%%%%%%%%%%%%%%%%%%%%%%

%% Energy, Momenta
\def\Ebeam {\ensuremath{E_{\mbox{\tiny BEAM}}}\xspace}
\def\sqs   {\ensuremath{\protect\sqrt{s}}\xspace}

\def\ptot       {\mbox{$p$}\xspace}
\def\pt         {\mbox{$p_{\mathrm{ T}}$}\xspace}
\def\et         {\mbox{$E_{\mathrm{ T}}$}\xspace}
\def\mt         {\mbox{$M_{\mathrm{ T}}$}\xspace}
\def\dpp        {\ensuremath{\Delta p/p}\xspace}
\def\msq        {\ensuremath{m^2}\xspace}
\newcommand{\dedx}{\ensuremath{\mathrm{d}\hspace{-0.1em}E/\mathrm{d}x}\xspace}

%% PID

\def\dllkpi     {\ensuremath{\mathrm{DLL}_{\kaon\pion}}\xspace}
\def\dllppi     {\ensuremath{\mathrm{DLL}_{\proton\pion}}\xspace}
\def\dllepi     {\ensuremath{\mathrm{DLL}_{\electron\pion}}\xspace}
\def\dllmupi    {\ensuremath{\mathrm{DLL}_{\muon\pi}}\xspace}

%% Geometry
%%%\def\mphi       {\mbox{$\phi$}\xspace}
%%%\def\mtheta     {\mbox{$\theta$}\xspace}
%%%\def\ctheta     {\mbox{$\cos\theta$}\xspace}
%%%\def\stheta     {\mbox{$\sin\theta$}\xspace}
%%%\def\ttheta     {\mbox{$\tan\theta$}\xspace}

\def\degrees{\ensuremath{^{\circ}}\xspace}
\def\krad {\ensuremath{\mathrm{ \,krad}}\xspace}
\def\mrad{\ensuremath{\mathrm{ \,mrad}}\xspace}
\def\rad{\ensuremath{\mathrm{ \,rad}}\xspace}

%% Accelerator
\def\betastar {\ensuremath{\beta^*}}
\newcommand{\lum} {\ensuremath{\mathcal{L}}\xspace}
\newcommand{\intlum}[1]{\ensuremath{\int\lum=#1}\xspace}  % {2 \,\invfb}

%%%%%%%%%%%%%%%%%%%%%%%%%%%%%%%%%%%%%%%%%%%%%%%%%%%%%%%%%%%%%%%%%%%%
% Software
%%%%%%%%%%%%%%%%%%%%%%%%%%%%%%%%%%%%%%%%%%%%%%%%%%%%%%%%%%%%%%%%%%%%

%% Programs
%%%\def\ansys      {\mbox{\textsc{Ansys}}\xspace}
\def\bcvegpy    {\mbox{\textsc{Bcvegpy}}\xspace}
\def\boole      {\mbox{\textsc{Boole}}\xspace}
\def\brunel     {\mbox{\textsc{Brunel}}\xspace}
\def\davinci    {\mbox{\textsc{DaVinci}}\xspace}
\def\dirac      {\mbox{\textsc{Dirac}}\xspace}
%%%\def\erasmus    {\mbox{\textsc{Erasmus}}\xspace}
\def\evtgen     {\mbox{\textsc{EvtGen}}\xspace}
\def\fewz       {\mbox{\textsc{Fewz}}\xspace}
\def\fluka      {\mbox{\textsc{Fluka}}\xspace}
\def\ganga      {\mbox{\textsc{Ganga}}\xspace}
%%%\def\garfield   {\mbox{\textsc{Garfield}}\xspace}
\def\gaudi      {\mbox{\textsc{Gaudi}}\xspace}
\def\gauss      {\mbox{\textsc{Gauss}}\xspace}
\def\geant      {\mbox{\textsc{Geant4}}\xspace}
\def\hepmc      {\mbox{\textsc{HepMC}}\xspace}
\def\herwig     {\mbox{\textsc{Herwig}}\xspace}
\def\moore      {\mbox{\textsc{Moore}}\xspace}
\def\neurobayes {\mbox{\textsc{NeuroBayes}}\xspace}
\def\photos     {\mbox{\textsc{Photos}}\xspace}
\def\powheg     {\mbox{\textsc{Powheg}}\xspace}
%%%\def\pyroot     {\mbox{\textsc{PyRoot}}\xspace}
\def\pythia     {\mbox{\textsc{Pythia}}\xspace}
\def\resbos     {\mbox{\textsc{ResBos}}\xspace}
\def\roofit     {\mbox{\textsc{RooFit}}\xspace}
\def\root       {\mbox{\textsc{Root}}\xspace}
\def\spice      {\mbox{\textsc{Spice}}\xspace}
%%%\def\tosca      {\mbox{\textsc{Tosca}}\xspace}
\def\urania     {\mbox{\textsc{Urania}}\xspace}

%% Languages
\def\cpp        {\mbox{\textsc{C\raisebox{0.1em}{{\footnotesize{++}}}}}\xspace}
%%%\def\python     {\mbox{\textsc{Python}}\xspace}
\def\ruby       {\mbox{\textsc{Ruby}}\xspace}
\def\fortran    {\mbox{\textsc{Fortran}}\xspace}
\def\svn        {\mbox{\textsc{SVN}}\xspace}

%% Data processing
\def\kbytes     {\ensuremath{{\mathrm{ \,kbytes}}}\xspace}
\def\kbsps      {\ensuremath{{\mathrm{ \,kbytes/s}}}\xspace}
\def\kbits      {\ensuremath{{\mathrm{ \,kbits}}}\xspace}
\def\kbsps      {\ensuremath{{\mathrm{ \,kbits/s}}}\xspace}
\def\mbsps      {\ensuremath{{\mathrm{ \,Mbits/s}}}\xspace}
\def\mbytes     {\ensuremath{{\mathrm{ \,Mbytes}}}\xspace}
\def\mbps       {\ensuremath{{\mathrm{ \,Mbyte/s}}}\xspace}
\def\mbsps      {\ensuremath{{\mathrm{ \,Mbytes/s}}}\xspace}
\def\gbsps      {\ensuremath{{\mathrm{ \,Gbits/s}}}\xspace}
\def\gbytes     {\ensuremath{{\mathrm{ \,Gbytes}}}\xspace}
\def\gbsps      {\ensuremath{{\mathrm{ \,Gbytes/s}}}\xspace}
\def\tbytes     {\ensuremath{{\mathrm{ \,Tbytes}}}\xspace}
\def\tbpy       {\ensuremath{{\mathrm{ \,Tbytes/yr}}}\xspace}

\def\dst        {DST\xspace}

%%%%%%%%%%%%%%%%%%%%%%%%%%%
% Detector related
%%%%%%%%%%%%%%%%%%%%%%%%%%%

%% Detector technologies
\def\nonn {\ensuremath{\mathrm{{ \mathit{n^+}} \mbox{-} on\mbox{-}{ \mathit{n}}}}\xspace}
\def\ponn {\ensuremath{\mathrm{{ \mathit{p^+}} \mbox{-} on\mbox{-}{ \mathit{n}}}}\xspace}
\def\nonp {\ensuremath{\mathrm{{ \mathit{n^+}} \mbox{-} on\mbox{-}{ \mathit{p}}}}\xspace}
\def\cvd  {CVD\xspace}
\def\mwpc {MWPC\xspace}
\def\gem  {GEM\xspace}

%% Detector components, electronics
\def\tell1  {TELL1\xspace}
\def\ukl1   {UKL1\xspace}
\def\beetle {Beetle\xspace}
\def\otis   {OTIS\xspace}
\def\croc   {CROC\xspace}
\def\carioca {CARIOCA\xspace}
\def\dialog {DIALOG\xspace}
\def\sync   {SYNC\xspace}
\def\cardiac {CARDIAC\xspace}
\def\gol    {GOL\xspace}
\def\vcsel  {VCSEL\xspace}
\def\ttc    {TTC\xspace}
\def\ttcrx  {TTCrx\xspace}
\def\hpd    {HPD\xspace}
\def\pmt    {PMT\xspace}
\def\specs  {SPECS\xspace}
\def\elmb   {ELMB\xspace}
\def\fpga   {FPGA\xspace}
\def\plc    {PLC\xspace}
\def\rasnik {RASNIK\xspace}
\def\elmb   {ELMB\xspace}
\def\can    {CAN\xspace}
\def\lvds   {LVDS\xspace}
\def\ntc    {NTC\xspace}
\def\adc    {ADC\xspace}
\def\led    {LED\xspace}
\def\ccd    {CCD\xspace}
\def\hv     {HV\xspace}
\def\lv     {LV\xspace}
\def\pvss   {PVSS\xspace}
\def\cmos   {CMOS\xspace}
\def\fifo   {FIFO\xspace}
\def\ccpc   {CCPC\xspace}

%% Chemical symbols
\def\cfourften     {\ensuremath{\mathrm{ C_4 F_{10}}}\xspace}
\def\cffour        {\ensuremath{\mathrm{ CF_4}}\xspace}
\def\cotwo         {\ensuremath{\mathrm{ CO_2}}\xspace} 
\def\csixffouteen  {\ensuremath{\mathrm{ C_6 F_{14}}}\xspace} 
\def\mgftwo     {\ensuremath{\mathrm{ Mg F_2}}\xspace} 
\def\siotwo     {\ensuremath{\mathrm{ SiO_2}}\xspace} 

%%%%%%%%%%%%%%%
% Special Text 
%%%%%%%%%%%%%%%
\newcommand{\eg}{\mbox{\itshape e.g.}\xspace}
\newcommand{\ie}{\mbox{\itshape i.e.}\xspace}
\newcommand{\etal}{\mbox{\itshape et al.}\xspace}
\newcommand{\etc}{\mbox{\itshape etc.}\xspace}
\newcommand{\cf}{\mbox{\itshape cf.}\xspace}
\newcommand{\ffp}{\mbox{\itshape ff.}\xspace}
\newcommand{\vs}{\mbox{\itshape vs.}\xspace}
 % Add in the predefined LHCb symbols
%% THis file contains all the default packages and modifications for
% LHCb formatting
%
\def\epsAcc {\ensuremath{\eps_{\rm acc}}\xspace}
\def\epsRec {\ensuremath{\eps_{\rm rec}}\xspace}
\def\epsTrig {\ensuremath{\eps_{\rm trig}}\xspace}
\def\epsPID {\ensuremath{\eps_{\rm PID}}\xspace}
\def\nsig {\ensuremath{n_{\rm sig}}\xspace}
\def\Ncor {\ensuremath{N^{\rm cor}}\xspace}
\def\ylab{\ensuremath{y_{\rm lab}}\xspace}

% used to align tables and equations
\newcommand{\xx}{\ensuremath{\kern 0.5em }}
\def\y {\ensuremath{y}\xspace}
\def\dy {\ensuremath{\deriv\y}\xspace}
\def\Dy {\ensuremath{\Delta\y}\xspace}
\def\dpt {\ensuremath{\deriv\pt}\xspace}
\def\Dpt {\ensuremath{\Delta\pt}\xspace}
\def\dsigma {\ensuremath{\deriv\sigma}\xspace}
\def\inter{\ensuremath{{\rm inter}}\xspace}
\def\Br {\ensuremath{{Br}}\xspace}
\def\NsigLb {\ensuremath{N_{\rm sig}^{\Lb}}\xspace}
\def\NsigBdb {\ensuremath{N_{\rm sig}^{\Bdb}}\xspace}
\def\epsLb {\ensuremath{\eps_{\rm tot}^{\Lb}}\xspace}
\def\epsBdb {\ensuremath{\eps_{\rm tot}^{\Bdb}}\xspace}
\def\RLbBdb{\ensuremath{R_{\Lb/\Bdb}}\xspace}
\def\fLbd{\ensuremath{f_{\Lb/d}}\xspace}
%\def\fLbd{\ensuremath{f_{\Lb}/f_d}\xspace}
\def\fLbud{\ensuremath{f_{\Lb}/(f_u+f_d)}\xspace}
\def\apd{\ensuremath{a_{\rm p+d}}\xspace}
\def\aprod{\ensuremath{a_{\rm prod}}\xspace}
\def\adecay{\ensuremath{a_{\rm decay}}\xspace}
\def\aDproton{\ensuremath{a_{\rm D}^{p}}\xspace}
\def\aDKaon{\ensuremath{a_{\rm D}^{K}}\xspace}
\def\aPID{\ensuremath{a_{\rm PID}}\xspace}
\def\Araw{\ensuremath{A_{\rm raw}}\xspace}
\def\araw{\ensuremath{a_{\rm raw}}\xspace}

\newcommand{\Lbpk}{\ensuremath{\Lb\to\jpsi\proton\Km}\xspace}
\newcommand{\antiLbpk}{\ensuremath{\Lbbar\to\jpsi\antiproton\Kp}\xspace}
\newcommand{\Bpik}{\ensuremath{\Bdb\to\jpsi\Kstarzb}\xspace}
\newcommand{\LbLcmunuX}{\ensuremath{\Lb\to\Lc\mun\neumb\PX}\xspace}
\newcommand{\LbLcpi}{\ensuremath{\Lb\to\Lc\pim}\xspace}
\newcommand{\LbJpsiLambda}{\ensuremath{\Lb\to\jpsi\Lz}\xspace}
\newcommand{\LbJpsippi}{\ensuremath{\Lb\to\jpsi\proton\pim}\xspace}
\newcommand{\fLbB}{\ensuremath{f(\Lb)/f(\Bdb)}\xspace}
\newcommand{\ccs}{\ensuremath{\cquark\cquarkbar\squark}\xspace}
\newcommand{\Jpsimumu}{\ensuremath{\jpsi\to\mumu}\xspace}
\newcommand{\psimumu}{\ensuremath{\psitwos\to\mumu}\xspace}
\newcommand{\KstarzbKpi}{\ensuremath{\Kstarzb\to\Km\pip}\xspace}
\newcommand{\BdbDpi}{\ensuremath{\Bdb\to\Dp\pim}\xspace}

% results 
\newcommand{\OneSinpA}{\ensuremath{380\pm\, 35\pm\, 19\,{\rm \nb}}}
\newcommand{\OneSinAp}{\ensuremath{295\pm\, 56\pm\, 27\,{\rm \nb}}}
\newcommand{\OneSinpAc}{\ensuremath{211\pm\, 23\pm\, 11\,{\rm \nb}}}
\newcommand{\OneSinApc}{\ensuremath{282\pm\, 53\pm\, 23\,{\rm \nb}}}
\newcommand{\TwoSinpA}{\ensuremath{\xx75\pm\, 19\pm\, \xx5\,{\rm \nb}}}
\newcommand{\ThreeSinpA}{\ensuremath{\xx27\pm\, 16\pm\, \xx4\,{\rm \nb}}}
\newcommand{\TwoSinAp}{\ensuremath{\xx81\pm\, 39\pm\, 17\,{\rm \nb}}}
\newcommand{\ThreeSinAp}{\ensuremath{\xx\xx5\pm\, 26\pm\, \xx5\,{\rm \nb}}}

%\newcommand{\TwoSinpA}{\ensuremath{\xx83\pm\, 19\pm\, \xx6\,{\rm \nb}}}
%\newcommand{\ThreeSinpA}{\ensuremath{\xx25\pm\, 15\pm\, \xx3\,{\rm \nb}}}
%\newcommand{\TwoSinAp}{\ensuremath{\xx67\pm\, 39\pm\, 14\,{\rm \nb}}}
%\newcommand{\ThreeSinAp}{\ensuremath{\xx16\pm\, 32\pm\, 14\,{\rm \nb}}}

% some definition for pPb collisions
\def\pp {\ensuremath{pp}\xspace}
\def\pPb {\ensuremath{p\mathrm{Pb}}\xspace}
\def\pA {\ensuremath{p\mathrm{A}}\xspace}
\def\dAu {\ensuremath{d\mathrm{Au}}\xspace}
\def\PbPb {\ensuremath{\mathrm{PbPb}}\xspace}
\def\sPlot{\mbox{\em sPlot}\xspace}
\def\sWeight{\mbox{\em sWeight}\xspace}
\def\sNN {\ensuremath{s_{\mbox{\tiny{\it NN}}}}\xspace}
\def\sNNtitle {\ensuremath{s_{\mbox{\small{\it NN}}}}\xspace}
\def\sqrtsNN {\ensuremath{\sqrt{\sNN}}\xspace}
\def\RpPb{\ensuremath{R_{p\mathrm{Pb}}}\xspace}
\def\RFB{\ensuremath{R_{\mbox{\tiny{FB}}}}\xspace}


\def\lone   {L0\xspace}
\def\hlt    {HLT\xspace}
\def\hltone {HLT1\xspace}
\def\hlttwo {HLT2\xspace}

\newcommand{\p}[1]{\ensuremath{\frac{\partial}{\partial{#1}} }}

\usepackage{multirow} % for complicated table
\usepackage{booktabs} % for complicated table
\usepackage{rotating}

\newcommand{\tabincell}[2]{\begin{tabular}{@{}#1@{}}#2\end{tabular}}


% Make this the last packages you include before the \begin{document}
\usepackage{cite} % Allows for ranges in citations
\usepackage{mciteplus}
%% THis file contains all the default packages and modifications for
%% LHCb formatting
%
%%% %%%%%%%%%%%%%%%%%%
%%%  Page formatting
%%% %%%%%%%%%%%%%%%%%%
%\textheight=230mm
%\textwidth=160mm
%\oddsidemargin=7mm
%\evensidemargin=-10mm
%\topmargin=-10mm
%\headsep=20mm
%\columnsep=5mm
%\addtolength{\belowcaptionskip}{0.5em}
%
%\renewcommand{\textfraction}{0.01}
%\renewcommand{\floatpagefraction}{0.99}
%\renewcommand{\topfraction}{0.9}
%\renewcommand{\bottomfraction}{0.9}
%
%
%\setlength{\hoffset}{-2cm}
%\setlength{\voffset}{-2cm}
%% Page defaults ...
%\topmargin=0.5cm
%\oddsidemargin=2.5cm
%\textwidth=16cm
%\textheight=22cm
%% Allow the page size to vary a bit ...
%\raggedbottom
%% To avoid Latex to be too fussy with line breaking ...
%\sloppy
%
%%% %%%%%%%%%%%%%%%%%%%%%%%
%%% Packages to be used
%%% %%%%%%%%%%%%%%%%%%%%%%% 
%\usepackage{geometry}
%\geometry{left=5cm,right=2.5cm,top=5cm,bottom=5cm}
%\usepackage{indentfirst}
%
%\usepackage{microtype}
%\usepackage{lineno}  % for line numbering during review
%\usepackage{xspace} % To avoid problems with missing or double spaces after
%                    % predefined symbold
%\usepackage{caption} %these three command get the figure and table captions automatically small
%\renewcommand{\captionfont}{\small}
%\renewcommand{\captionlabelfont}{\small}
%
%%% Graphics
%\usepackage{graphicx}  % to include figures (can also use other packages)
%\usepackage{color}
%\usepackage{colortbl}
%\graphicspath{{./figs/}} % Make Latex search fig subdir for figures
%
%%% Math
%\usepackage{amsmath} % Adds a large collection of math symbols
%\usepackage{amssymb}
%\usepackage{amsfonts}
%\usepackage{upgreek} % Adds in support for greek letters in roman typeset
%
%%% fix to allow peaceful coexistence of line numbering and
%%% mathematical objects
%%% http://www.latex-community.org/forum/viewtopic.php?f=5&t=163
%%%
%\newcommand*\patchAmsMathEnvironmentForLineno[1]{%
%\expandafter\let\csname old#1\expandafter\endcsname\csname #1\endcsname
%\expandafter\let\csname oldend#1\expandafter\endcsname\csname
%end#1\endcsname
% \renewenvironment{#1}%
%   {\linenomath\csname old#1\endcsname}%
%   {\csname oldend#1\endcsname\endlinenomath}%
%}
%\newcommand*\patchBothAmsMathEnvironmentsForLineno[1]{%
%  \patchAmsMathEnvironmentForLineno{#1}%
%  \patchAmsMathEnvironmentForLineno{#1*}%
%}
%\AtBeginDocument{%
%\patchBothAmsMathEnvironmentsForLineno{equation}%
%\patchBothAmsMathEnvironmentsForLineno{align}%
%\patchBothAmsMathEnvironmentsForLineno{flalign}%
%\patchBothAmsMathEnvironmentsForLineno{alignat}%
%\patchBothAmsMathEnvironmentsForLineno{gather}%
%\patchBothAmsMathEnvironmentsForLineno{multline}%
%\patchBothAmsMathEnvironmentsForLineno{eqnarray}%
%}
%
%% Get hyperlinks to captions and in references.
%% These do not work with revtex. Use "hypertext" as class option instead.
%\usepackage{hyperref}    % Hyperlinks in references
%\usepackage[all]{hypcap} % Internal hyperlinks to floats.
%
%%%% $Id: lhcb-symbols-def.tex 78711 2015-08-06 07:54:32Z apuignav $
%%% ======================================================================
%%% Purpose: Standard LHCb aliases
%%% Author: Originally Ulrik Egede, adapted by Tomasz Skwarnicki for templates,
%%% rewritten by Chris Parkes
%%% Maintainer : Ulrik Egede (2010 - 2012)
%%% Maintainer : Rolf Oldeman (2012 - 2014)
%%% =======================================================================

%%% To use this file outside the normal LHCb document environment, the
%%% following should be added in a preamble (before \begin{document}
%%%
%%%\usepackage{ifthen} 
%%%\newboolean{uprightparticles}
%%%\setboolean{uprightparticles}{false} %Set true for upright particle symbols
\usepackage{xspace} 
\usepackage{upgreek}

%%%%%%%%%%%%%%%%%%%%%%%%%%%%%%%%%%%%%%%%%%%%%%%%%%%%%%%%%%%%
%%%
%%% The following is to ensure that the template automatically can process
%%% this file.
%%%
%%% Add comments with at least three %%% preceding.
%%% Add new sections with one % preceding
%%% Add new subsections with two %% preceding
%%%%%%%%%%%%%%%%%%%%%%%%%%%%%%%%%%%%%%%%%%%%%%%%%%%%%%%%%%%%

%%%%%%%%%%%%%
% Experiments
%%%%%%%%%%%%%
\def\lhcb {\mbox{LHCb}\xspace}
\def\atlas  {\mbox{ATLAS}\xspace}
\def\cms    {\mbox{CMS}\xspace}
\def\alice  {\mbox{ALICE}\xspace}
\def\babar  {\mbox{BaBar}\xspace}
\def\belle  {\mbox{Belle}\xspace}
\def\cleo   {\mbox{CLEO}\xspace}
\def\cdf    {\mbox{CDF}\xspace}
\def\dzero  {\mbox{D0}\xspace}
\def\aleph  {\mbox{ALEPH}\xspace}
\def\delphi {\mbox{DELPHI}\xspace}
\def\opal   {\mbox{OPAL}\xspace}
\def\lthree {\mbox{L3}\xspace}
\def\sld    {\mbox{SLD}\xspace}
%%%\def\argus  {\mbox{ARGUS}\xspace}
%%%\def\uaone  {\mbox{UA1}\xspace}
%%%\def\uatwo  {\mbox{UA2}\xspace}
%%%\def\ux85 {\mbox{UX85}\xspace}
\def\cern {\mbox{CERN}\xspace}
\def\lhc    {\mbox{LHC}\xspace}
\def\lep    {\mbox{LEP}\xspace}
\def\tevatron {Tevatron\xspace}

%% LHCb sub-detectors and sub-systems

%%%\def\pu     {PU\xspace}
\def\velo   {VELO\xspace}
\def\rich   {RICH\xspace}
\def\richone {RICH1\xspace}
\def\richtwo {RICH2\xspace}
\def\ttracker {TT\xspace}
\def\intr   {IT\xspace}
\def\st     {ST\xspace}
\def\ot     {OT\xspace}
%%%\def\Tone   {T1\xspace}
%%%\def\Ttwo   {T2\xspace}
%%%\def\Tthree {T3\xspace}
%%%\def\Mone   {M1\xspace}
%%%\def\Mtwo   {M2\xspace}
%%%\def\Mthree {M3\xspace}
%%%\def\Mfour  {M4\xspace}
%%%\def\Mfive  {M5\xspace}
\def\spd    {SPD\xspace}
\def\presh  {PS\xspace}
\def\ecal   {ECAL\xspace}
\def\hcal   {HCAL\xspace}
%%%\def\bcm    {BCM\xspace}
\def\MagUp {\mbox{\em Mag\kern -0.05em Up}\xspace}
\def\MagDown {\mbox{\em MagDown}\xspace}

\def\ode    {ODE\xspace}
\def\daq    {DAQ\xspace}
\def\tfc    {TFC\xspace}
\def\ecs    {ECS\xspace}
\def\lone   {L0\xspace}
\def\hlt    {HLT\xspace}
\def\hltone {HLT1\xspace}
\def\hlttwo {HLT2\xspace}

%%% Upright (not slanted) Particles

\ifthenelse{\boolean{uprightparticles}}%
{\def\Palpha      {\ensuremath{\upalpha}\xspace}
 \def\Pbeta       {\ensuremath{\upbeta}\xspace}
 \def\Pgamma      {\ensuremath{\upgamma}\xspace}                 
 \def\Pdelta      {\ensuremath{\updelta}\xspace}                 
 \def\Pepsilon    {\ensuremath{\upepsilon}\xspace}                 
 \def\Pvarepsilon {\ensuremath{\upvarepsilon}\xspace}                 
 \def\Pzeta       {\ensuremath{\upzeta}\xspace}                 
 \def\Peta        {\ensuremath{\upeta}\xspace}                 
 \def\Ptheta      {\ensuremath{\uptheta}\xspace}                 
 \def\Pvartheta   {\ensuremath{\upvartheta}\xspace}                 
 \def\Piota       {\ensuremath{\upiota}\xspace}                 
 \def\Pkappa      {\ensuremath{\upkappa}\xspace}                 
 \def\Plambda     {\ensuremath{\uplambda}\xspace}                 
 \def\Pmu         {\ensuremath{\upmu}\xspace}                 
 \def\Pnu         {\ensuremath{\upnu}\xspace}                 
 \def\Pxi         {\ensuremath{\upxi}\xspace}                 
 \def\Ppi         {\ensuremath{\uppi}\xspace}                 
 \def\Pvarpi      {\ensuremath{\upvarpi}\xspace}                 
 \def\Prho        {\ensuremath{\uprho}\xspace}                 
 \def\Pvarrho     {\ensuremath{\upvarrho}\xspace}                 
 \def\Ptau        {\ensuremath{\uptau}\xspace}                 
 \def\Pupsilon    {\ensuremath{\upupsilon}\xspace}                 
 \def\Pphi        {\ensuremath{\upphi}\xspace}                 
 \def\Pvarphi     {\ensuremath{\upvarphi}\xspace}                 
 \def\Pchi        {\ensuremath{\upchi}\xspace}                 
 \def\Ppsi        {\ensuremath{\uppsi}\xspace}                 
 \def\Pomega      {\ensuremath{\upomega}\xspace}                 

 \def\PDelta      {\ensuremath{\Delta}\xspace}                 
 \def\PXi      {\ensuremath{\Xi}\xspace}                 
 \def\PLambda      {\ensuremath{\Lambda}\xspace}                 
 \def\PSigma      {\ensuremath{\Sigma}\xspace}                 
 \def\POmega      {\ensuremath{\Omega}\xspace}                 
 \def\PUpsilon      {\ensuremath{\Upsilon}\xspace}                 
 
 %\mathchardef\Deltares="7101
 %\mathchardef\Xi="7104
 %\mathchardef\Lambda="7103
 %\mathchardef\Sigma="7106
 %\mathchardef\Omega="710A


 \def\PA      {\ensuremath{\mathrm{A}}\xspace}                 
 \def\PB      {\ensuremath{\mathrm{B}}\xspace}                 
 \def\PC      {\ensuremath{\mathrm{C}}\xspace}                 
 \def\PD      {\ensuremath{\mathrm{D}}\xspace}                 
 \def\PE      {\ensuremath{\mathrm{E}}\xspace}                 
 \def\PF      {\ensuremath{\mathrm{F}}\xspace}                 
 \def\PG      {\ensuremath{\mathrm{G}}\xspace}                 
 \def\PH      {\ensuremath{\mathrm{H}}\xspace}                 
% \def\PI      {\ensuremath{\mathrm{I}}\xspace}                 
 \def\PJ      {\ensuremath{\mathrm{J}}\xspace}                 
 \def\PK      {\ensuremath{\mathrm{K}}\xspace}                 
 \def\PL      {\ensuremath{\mathrm{L}}\xspace}                 
 \def\PM      {\ensuremath{\mathrm{M}}\xspace}                 
 \def\PN      {\ensuremath{\mathrm{N}}\xspace}                 
 \def\PO      {\ensuremath{\mathrm{O}}\xspace}                 
 \def\PP      {\ensuremath{\mathrm{P}}\xspace}                 
 \def\PQ      {\ensuremath{\mathrm{Q}}\xspace}                 
 \def\PR      {\ensuremath{\mathrm{R}}\xspace}                 
 \def\PS      {\ensuremath{\mathrm{S}}\xspace}                 
 \def\PT      {\ensuremath{\mathrm{T}}\xspace}                 
 \def\PU      {\ensuremath{\mathrm{U}}\xspace}                 
 \def\PV      {\ensuremath{\mathrm{V}}\xspace}                 
 \def\PW      {\ensuremath{\mathrm{W}}\xspace}                 
 \def\PX      {\ensuremath{\mathrm{X}}\xspace}                 
 \def\PY      {\ensuremath{\mathrm{Y}}\xspace}                 
 \def\PZ      {\ensuremath{\mathrm{Z}}\xspace}                 
 \def\Pa      {\ensuremath{\mathrm{a}}\xspace}                 
 \def\Pb      {\ensuremath{\mathrm{b}}\xspace}                 
 \def\Pc      {\ensuremath{\mathrm{c}}\xspace}                 
 \def\Pd      {\ensuremath{\mathrm{d}}\xspace}                 
 \def\Pe      {\ensuremath{\mathrm{e}}\xspace}                 
 \def\Pf      {\ensuremath{\mathrm{f}}\xspace}                 
 \def\Pg      {\ensuremath{\mathrm{g}}\xspace}                 
 \def\Ph      {\ensuremath{\mathrm{h}}\xspace}                 
% \def\Pi      {\ensuremath{\mathrm{i}}\xspace}                 
 \def\Pj      {\ensuremath{\mathrm{j}}\xspace}                 
 \def\Pk      {\ensuremath{\mathrm{k}}\xspace}                 
 \def\Pl      {\ensuremath{\mathrm{l}}\xspace}                 
 \def\Pm      {\ensuremath{\mathrm{m}}\xspace}                 
 \def\Pn      {\ensuremath{\mathrm{n}}\xspace}                 
 \def\Po      {\ensuremath{\mathrm{o}}\xspace}                 
 \def\Pp      {\ensuremath{\mathrm{p}}\xspace}                 
 \def\Pq      {\ensuremath{\mathrm{q}}\xspace}                 
 \def\Pr      {\ensuremath{\mathrm{r}}\xspace}                 
 \def\Ps      {\ensuremath{\mathrm{s}}\xspace}                 
 \def\Pt      {\ensuremath{\mathrm{t}}\xspace}                 
 \def\Pu      {\ensuremath{\mathrm{u}}\xspace}                 
 \def\Pv      {\ensuremath{\mathrm{v}}\xspace}                 
 \def\Pw      {\ensuremath{\mathrm{w}}\xspace}                 
 \def\Px      {\ensuremath{\mathrm{x}}\xspace}                 
 \def\Py      {\ensuremath{\mathrm{y}}\xspace}                 
 \def\Pz      {\ensuremath{\mathrm{z}}\xspace}                 
}
{\def\Palpha      {\ensuremath{\alpha}\xspace}
 \def\Pbeta       {\ensuremath{\beta}\xspace}
 \def\Pgamma      {\ensuremath{\gamma}\xspace}                 
 \def\Pdelta      {\ensuremath{\delta}\xspace}                 
 \def\Pepsilon    {\ensuremath{\epsilon}\xspace}                 
 \def\Pvarepsilon {\ensuremath{\varepsilon}\xspace}                 
 \def\Pzeta       {\ensuremath{\zeta}\xspace}                 
 \def\Peta        {\ensuremath{\eta}\xspace}                 
 \def\Ptheta      {\ensuremath{\theta}\xspace}                 
 \def\Pvartheta   {\ensuremath{\vartheta}\xspace}                 
 \def\Piota       {\ensuremath{\iota}\xspace}                 
 \def\Pkappa      {\ensuremath{\kappa}\xspace}                 
 \def\Plambda     {\ensuremath{\lambda}\xspace}                 
 \def\Pmu         {\ensuremath{\mu}\xspace}                 
 \def\Pnu         {\ensuremath{\nu}\xspace}                 
 \def\Pxi         {\ensuremath{\xi}\xspace}                 
 \def\Ppi         {\ensuremath{\pi}\xspace}                 
 \def\Pvarpi      {\ensuremath{\varpi}\xspace}                 
 \def\Prho        {\ensuremath{\rho}\xspace}                 
 \def\Pvarrho     {\ensuremath{\varrho}\xspace}                 
 \def\Ptau        {\ensuremath{\tau}\xspace}                 
 \def\Pupsilon    {\ensuremath{\upsilon}\xspace}                 
 \def\Pphi        {\ensuremath{\phi}\xspace}                 
 \def\Pvarphi     {\ensuremath{\varphi}\xspace}                 
 \def\Pchi        {\ensuremath{\chi}\xspace}                 
 \def\Ppsi        {\ensuremath{\psi}\xspace}                 
 \def\Pomega      {\ensuremath{\omega}\xspace}                 
 \mathchardef\PDelta="7101
 \mathchardef\PXi="7104
 \mathchardef\PLambda="7103
 \mathchardef\PSigma="7106
 \mathchardef\POmega="710A
 \mathchardef\PUpsilon="7107
 \def\PA      {\ensuremath{A}\xspace}                 
 \def\PB      {\ensuremath{B}\xspace}                 
 \def\PC      {\ensuremath{C}\xspace}                 
 \def\PD      {\ensuremath{D}\xspace}                 
 \def\PE      {\ensuremath{E}\xspace}                 
 \def\PF      {\ensuremath{F}\xspace}                 
 \def\PG      {\ensuremath{G}\xspace}                 
 \def\PH      {\ensuremath{H}\xspace}                 
 \def\PI      {\ensuremath{I}\xspace}                 
 \def\PJ      {\ensuremath{J}\xspace}                 
 \def\PK      {\ensuremath{K}\xspace}                 
 \def\PL      {\ensuremath{L}\xspace}                 
 \def\PM      {\ensuremath{M}\xspace}                 
 \def\PN      {\ensuremath{N}\xspace}                 
 \def\PO      {\ensuremath{O}\xspace}                 
 \def\PP      {\ensuremath{P}\xspace}                 
 \def\PQ      {\ensuremath{Q}\xspace}                 
 \def\PR      {\ensuremath{R}\xspace}                 
 \def\PS      {\ensuremath{S}\xspace}                 
 \def\PT      {\ensuremath{T}\xspace}                 
 \def\PU      {\ensuremath{U}\xspace}                 
 \def\PV      {\ensuremath{V}\xspace}                 
 \def\PW      {\ensuremath{W}\xspace}                 
 \def\PX      {\ensuremath{X}\xspace}                 
 \def\PY      {\ensuremath{Y}\xspace}                 
 \def\PZ      {\ensuremath{Z}\xspace}                 
 \def\Pa      {\ensuremath{a}\xspace}                 
 \def\Pb      {\ensuremath{b}\xspace}                 
 \def\Pc      {\ensuremath{c}\xspace}                 
 \def\Pd      {\ensuremath{d}\xspace}                 
 \def\Pe      {\ensuremath{e}\xspace}                 
 \def\Pf      {\ensuremath{f}\xspace}                 
 \def\Pg      {\ensuremath{g}\xspace}                 
 \def\Ph      {\ensuremath{h}\xspace}                 
% \def\Pi      {\ensuremath{i}\xspace}                 
 \def\Pj      {\ensuremath{j}\xspace}                 
 \def\Pk      {\ensuremath{k}\xspace}                 
 \def\Pl      {\ensuremath{l}\xspace}                 
 \def\Pm      {\ensuremath{m}\xspace}                 
 \def\Pn      {\ensuremath{n}\xspace}                 
 \def\Po      {\ensuremath{o}\xspace}                 
 \def\Pp      {\ensuremath{p}\xspace}                 
 \def\Pq      {\ensuremath{q}\xspace}                 
 \def\Pr      {\ensuremath{r}\xspace}                 
 \def\Ps      {\ensuremath{s}\xspace}                 
 \def\Pt      {\ensuremath{t}\xspace}                 
 \def\Pu      {\ensuremath{u}\xspace}                 
 \def\Pv      {\ensuremath{v}\xspace}                 
 \def\Pw      {\ensuremath{w}\xspace}                 
 \def\Px      {\ensuremath{x}\xspace}                 
 \def\Py      {\ensuremath{y}\xspace}                 
 \def\Pz      {\ensuremath{z}\xspace}                 
}

%%%%%%%%%%%%%%%%%%%%%%%%%%%%%%%%%%%%%%%%%%%%%%%
% Particles
\makeatletter
\ifcase \@ptsize \relax% 10pt
  \newcommand{\miniscule}{\@setfontsize\miniscule{4}{5}}% \tiny: 5/6
\or% 11pt
  \newcommand{\miniscule}{\@setfontsize\miniscule{5}{6}}% \tiny: 6/7
\or% 12pt
  \newcommand{\miniscule}{\@setfontsize\miniscule{5}{6}}% \tiny: 6/7
\fi
\makeatother


\DeclareRobustCommand{\optbar}[1]{\shortstack{{\miniscule (\rule[.5ex]{1.25em}{.18mm})}
  \\ [-.7ex] $#1$}}


%% Leptons

\let\emi\en
\def\electron   {{\ensuremath{\Pe}}\xspace}
\def\en         {{\ensuremath{\Pe^-}}\xspace}   % electron negative (\em is taken)
\def\ep         {{\ensuremath{\Pe^+}}\xspace}
\def\epm        {{\ensuremath{\Pe^\pm}}\xspace} 
\def\epem       {{\ensuremath{\Pe^+\Pe^-}}\xspace}
%%%\def\ee         {\ensuremath{\Pe^-\Pe^-}\xspace}

\def\muon       {{\ensuremath{\Pmu}}\xspace}
\def\mup        {{\ensuremath{\Pmu^+}}\xspace}
\def\mun        {{\ensuremath{\Pmu^-}}\xspace} % muon negative (\mum is taken)
\def\mumu       {{\ensuremath{\Pmu^+\Pmu^-}}\xspace}

\def\tauon      {{\ensuremath{\Ptau}}\xspace}
\def\taup       {{\ensuremath{\Ptau^+}}\xspace}
\def\taum       {{\ensuremath{\Ptau^-}}\xspace}
\def\tautau     {{\ensuremath{\Ptau^+\Ptau^-}}\xspace}

\def\lepton     {{\ensuremath{\ell}}\xspace}
\def\ellm       {{\ensuremath{\ell^-}}\xspace}
\def\ellp       {{\ensuremath{\ell^+}}\xspace}
%%%\def\ellell     {\ensuremath{\ell^+ \ell^-}\xspace}

\def\neu        {{\ensuremath{\Pnu}}\xspace}
\def\neub       {{\ensuremath{\overline{\Pnu}}}\xspace}
%%%\def\nuenueb    {\ensuremath{\neu\neub}\xspace}
\def\neue       {{\ensuremath{\neu_e}}\xspace}
\def\neueb      {{\ensuremath{\neub_e}}\xspace}
%%%\def\neueneueb  {\ensuremath{\neue\neueb}\xspace}
\def\neum       {{\ensuremath{\neu_\mu}}\xspace}
\def\neumb      {{\ensuremath{\neub_\mu}}\xspace}
%%%\def\neumneumb  {\ensuremath{\neum\neumb}\xspace}
\def\neut       {{\ensuremath{\neu_\tau}}\xspace}
\def\neutb      {{\ensuremath{\neub_\tau}}\xspace}
%%%\def\neutneutb  {\ensuremath{\neut\neutb}\xspace}
\def\neul       {{\ensuremath{\neu_\ell}}\xspace}
\def\neulb      {{\ensuremath{\neub_\ell}}\xspace}
%%%\def\neulneulb  {\ensuremath{\neul\neulb}\xspace}

%% Gauge bosons and scalars

\def\g      {{\ensuremath{\Pgamma}}\xspace}
\def\H      {{\ensuremath{\PH^0}}\xspace}
\def\Hp     {{\ensuremath{\PH^+}}\xspace}
\def\Hm     {{\ensuremath{\PH^-}}\xspace}
\def\Hpm    {{\ensuremath{\PH^\pm}}\xspace}
\def\W      {{\ensuremath{\PW}}\xspace}
\def\Wp     {{\ensuremath{\PW^+}}\xspace}
\def\Wm     {{\ensuremath{\PW^-}}\xspace}
\def\Wpm    {{\ensuremath{\PW^\pm}}\xspace}
\def\Z      {{\ensuremath{\PZ}}\xspace}

%% Quarks

\def\quark     {{\ensuremath{\Pq}}\xspace}
\def\quarkbar  {{\ensuremath{\overline \quark}}\xspace}
\def\qqbar     {{\ensuremath{\quark\quarkbar}}\xspace}
\def\uquark    {{\ensuremath{\Pu}}\xspace}
\def\uquarkbar {{\ensuremath{\overline \uquark}}\xspace}
\def\uubar     {{\ensuremath{\uquark\uquarkbar}}\xspace}
\def\dquark    {{\ensuremath{\Pd}}\xspace}
\def\dquarkbar {{\ensuremath{\overline \dquark}}\xspace}
\def\ddbar     {{\ensuremath{\dquark\dquarkbar}}\xspace}
\def\squark    {{\ensuremath{\Ps}}\xspace}
\def\squarkbar {{\ensuremath{\overline \squark}}\xspace}
\def\ssbar     {{\ensuremath{\squark\squarkbar}}\xspace}
\def\cquark    {{\ensuremath{\Pc}}\xspace}
\def\cquarkbar {{\ensuremath{\overline \cquark}}\xspace}
\def\ccbar     {{\ensuremath{\cquark\cquarkbar}}\xspace}
\def\bquark    {{\ensuremath{\Pb}}\xspace}
\def\bquarkbar {{\ensuremath{\overline \bquark}}\xspace}
\def\bbbar     {{\ensuremath{\bquark\bquarkbar}}\xspace}
\def\tquark    {{\ensuremath{\Pt}}\xspace}
\def\tquarkbar {{\ensuremath{\overline \tquark}}\xspace}
\def\ttbar     {{\ensuremath{\tquark\tquarkbar}}\xspace}

%% Light mesons

\def\hadron {{\ensuremath{\Ph}}\xspace}
\def\pion   {{\ensuremath{\Ppi}}\xspace}
\def\piz    {{\ensuremath{\pion^0}}\xspace}
\def\pizs   {{\ensuremath{\pion^0\mbox\,\mathrm{s}}}\xspace}
\def\pip    {{\ensuremath{\pion^+}}\xspace}
\def\pim    {{\ensuremath{\pion^-}}\xspace}
\def\pipm   {{\ensuremath{\pion^\pm}}\xspace}
\def\pimp   {{\ensuremath{\pion^\mp}}\xspace}

\def\rhomeson {{\ensuremath{\Prho}}\xspace}
\def\rhoz     {{\ensuremath{\rhomeson^0}}\xspace}
\def\rhop     {{\ensuremath{\rhomeson^+}}\xspace}
\def\rhom     {{\ensuremath{\rhomeson^-}}\xspace}
\def\rhopm    {{\ensuremath{\rhomeson^\pm}}\xspace}
\def\rhomp    {{\ensuremath{\rhomeson^\mp}}\xspace}

\def\kaon    {{\ensuremath{\PK}}\xspace}
%%% do NOT use ensuremath here
  \def\Kbar    {{\kern 0.2em\overline{\kern -0.2em \PK}{}}\xspace}
\def\Kb      {{\ensuremath{\Kbar}}\xspace}
\def\KorKbar    {\kern 0.18em\optbar{\kern -0.18em K}{}\xspace}
\def\Kz      {{\ensuremath{\kaon^0}}\xspace}
\def\Kzb     {{\ensuremath{\Kbar{}^0}}\xspace}
\def\Kp      {{\ensuremath{\kaon^+}}\xspace}
\def\Km      {{\ensuremath{\kaon^-}}\xspace}
\def\Kpm     {{\ensuremath{\kaon^\pm}}\xspace}
\def\Kmp     {{\ensuremath{\kaon^\mp}}\xspace}
\def\KS      {{\ensuremath{\kaon^0_{\mathrm{ \scriptscriptstyle S}}}}\xspace}
\def\KL      {{\ensuremath{\kaon^0_{\mathrm{ \scriptscriptstyle L}}}}\xspace}
\def\Kstarz  {{\ensuremath{\kaon^{*0}}}\xspace}
\def\Kstarzb {{\ensuremath{\Kbar{}^{*0}}}\xspace}
\def\Kstar   {{\ensuremath{\kaon^*}}\xspace}
\def\Kstarb  {{\ensuremath{\Kbar{}^*}}\xspace}
\def\Kstarp  {{\ensuremath{\kaon^{*+}}}\xspace}
\def\Kstarm  {{\ensuremath{\kaon^{*-}}}\xspace}
\def\Kstarpm {{\ensuremath{\kaon^{*\pm}}}\xspace}
\def\Kstarmp {{\ensuremath{\kaon^{*\mp}}}\xspace}

\newcommand{\etaz}{\ensuremath{\Peta}\xspace}
\newcommand{\etapr}{\ensuremath{\Peta^{\prime}}\xspace}
\newcommand{\phiz}{\ensuremath{\Pphi}\xspace}
\newcommand{\omegaz}{\ensuremath{\Pomega}\xspace}

%% Heavy mesons

%%% do NOT use ensuremath here
  \def\Dbar    {{\kern 0.2em\overline{\kern -0.2em \PD}{}}\xspace}
\def\D       {{\ensuremath{\PD}}\xspace}
\def\Db      {{\ensuremath{\Dbar}}\xspace}
\def\DorDbar    {\kern 0.18em\optbar{\kern -0.18em D}{}\xspace}
\def\Dz      {{\ensuremath{\D^0}}\xspace}
\def\Dzb     {{\ensuremath{\Dbar{}^0}}\xspace}
\def\Dp      {{\ensuremath{\D^+}}\xspace}
\def\Dm      {{\ensuremath{\D^-}}\xspace}
\def\Dpm     {{\ensuremath{\D^\pm}}\xspace}
\def\Dmp     {{\ensuremath{\D^\mp}}\xspace}
\def\Dstar   {{\ensuremath{\D^*}}\xspace}
\def\Dstarb  {{\ensuremath{\Dbar{}^*}}\xspace}
\def\Dstarz  {{\ensuremath{\D^{*0}}}\xspace}
\def\Dstarzb {{\ensuremath{\Dbar{}^{*0}}}\xspace}
\def\Dstarp  {{\ensuremath{\D^{*+}}}\xspace}
\def\Dstarm  {{\ensuremath{\D^{*-}}}\xspace}
\def\Dstarpm {{\ensuremath{\D^{*\pm}}}\xspace}
\def\Dstarmp {{\ensuremath{\D^{*\mp}}}\xspace}
\def\Ds      {{\ensuremath{\D^+_\squark}}\xspace}
\def\Dsp     {{\ensuremath{\D^+_\squark}}\xspace}
\def\Dsm     {{\ensuremath{\D^-_\squark}}\xspace}
\def\Dspm    {{\ensuremath{\D^{\pm}_\squark}}\xspace}
\def\Dsmp    {{\ensuremath{\D^{\mp}_\squark}}\xspace}
\def\Dss     {{\ensuremath{\D^{*+}_\squark}}\xspace}
\def\Dssp    {{\ensuremath{\D^{*+}_\squark}}\xspace}
\def\Dssm    {{\ensuremath{\D^{*-}_\squark}}\xspace}
\def\Dsspm   {{\ensuremath{\D^{*\pm}_\squark}}\xspace}
\def\Dssmp   {{\ensuremath{\D^{*\mp}_\squark}}\xspace}

\def\B       {{\ensuremath{\PB}}\xspace}
%%% do NOT use ensuremath here
\def\Bbar    {{\ensuremath{\kern 0.18em\overline{\kern -0.18em \PB}{}}}\xspace}
\def\Bb      {{\ensuremath{\Bbar}}\xspace}
\def\BorBbar    {\kern 0.18em\optbar{\kern -0.18em B}{}\xspace}
\def\Bz      {{\ensuremath{\B^0}}\xspace}
\def\Bzb     {{\ensuremath{\Bbar{}^0}}\xspace}
\def\Bu      {{\ensuremath{\B^+}}\xspace}
\def\Bub     {{\ensuremath{\B^-}}\xspace}
\def\Bp      {{\ensuremath{\Bu}}\xspace}
\def\Bm      {{\ensuremath{\Bub}}\xspace}
\def\Bpm     {{\ensuremath{\B^\pm}}\xspace}
\def\Bmp     {{\ensuremath{\B^\mp}}\xspace}
\def\Bd      {{\ensuremath{\B^0}}\xspace}
\def\Bs      {{\ensuremath{\B^0_\squark}}\xspace}
\def\Bsb     {{\ensuremath{\Bbar{}^0_\squark}}\xspace}
\def\Bdb     {{\ensuremath{\Bbar{}^0}}\xspace}
\def\Bc      {{\ensuremath{\B_\cquark^+}}\xspace}
\def\Bcp     {{\ensuremath{\B_\cquark^+}}\xspace}
\def\Bcm     {{\ensuremath{\B_\cquark^-}}\xspace}
\def\Bcpm    {{\ensuremath{\B_\cquark^\pm}}\xspace}

%% Onia

\def\jpsi     {{\ensuremath{{\PJ\mskip -3mu/\mskip -2mu\Ppsi\mskip 2mu}}}\xspace}
\def\psitwos  {{\ensuremath{\Ppsi{(2S)}}}\xspace}
\def\psiprpr  {{\ensuremath{\Ppsi(3770)}}\xspace}
\def\etac     {{\ensuremath{\Peta_\cquark}}\xspace}
\def\chiczero {{\ensuremath{\Pchi_{\cquark 0}}}\xspace}
\def\chicone  {{\ensuremath{\Pchi_{\cquark 1}}}\xspace}
\def\chictwo  {{\ensuremath{\Pchi_{\cquark 2}}}\xspace}
  %\mathchardef\Upsilon="7107
  \def\Y#1S{\ensuremath{\PUpsilon{(#1S)}}\xspace}% no space before {...}!
\def\OneS  {{\Y1S}}
\def\TwoS  {{\Y2S}}
\def\ThreeS{{\Y3S}}
\def\FourS {{\Y4S}}
\def\FiveS {{\Y5S}}

\def\chic  {{\ensuremath{\Pchi_{c}}}\xspace}

%% Baryons

\def\proton      {{\ensuremath{\Pp}}\xspace}
\def\antiproton  {{\ensuremath{\overline \proton}}\xspace}
\def\neutron     {{\ensuremath{\Pn}}\xspace}
\def\antineutron {{\ensuremath{\overline \neutron}}\xspace}
\def\Deltares    {{\ensuremath{\PDelta}}\xspace}
\def\Deltaresbar {{\ensuremath{\overline \Deltares}}\xspace}
\def\Xires       {{\ensuremath{\PXi}}\xspace}
\def\Xiresbar    {{\ensuremath{\overline \Xires}}\xspace}
\def\Lz          {{\ensuremath{\PLambda}}\xspace}
\def\Lbar        {{\ensuremath{\kern 0.1em\overline{\kern -0.1em\PLambda}}}\xspace}
\def\LorLbar    {\kern 0.18em\optbar{\kern -0.18em \PLambda}{}\xspace}
\def\Lambdares   {{\ensuremath{\PLambda}}\xspace}
\def\Lambdaresbar{{\ensuremath{\Lbar}}\xspace}
\def\Sigmares    {{\ensuremath{\PSigma}}\xspace}
\def\Sigmaresbar {{\ensuremath{\overline \Sigmares}}\xspace}
\def\Omegares    {{\ensuremath{\POmega}}\xspace}
\def\Omegaresbar {{\ensuremath{\overline \POmega}}\xspace}

%%% do NOT use ensuremath here
 % \def\Deltabar{\kern 0.25em\overline{\kern -0.25em \Deltares}{}\xspace}
 % \def\Sigbar{\kern 0.2em\overline{\kern -0.2em \Sigma}{}\xspace}
 % \def\Xibar{\kern 0.2em\overline{\kern -0.2em \Xi}{}\xspace}
 % \def\Obar{\kern 0.2em\overline{\kern -0.2em \Omega}{}\xspace}
 % \def\Nbar{\kern 0.2em\overline{\kern -0.2em N}{}\xspace}
 % \def\Xb{\kern 0.2em\overline{\kern -0.2em X}{}\xspace}

\def\Lb      {{\ensuremath{\Lz^0_\bquark}}\xspace}
\def\Lbbar   {{\ensuremath{\Lbar{}^0_\bquark}}\xspace}
\def\Lc      {{\ensuremath{\Lz^+_\cquark}}\xspace}
\def\Lcbar   {{\ensuremath{\Lbar{}^-_\cquark}}\xspace}
\def\Xib     {{\ensuremath{\Xires_\bquark}}\xspace}
\def\Xibz    {{\ensuremath{\Xires^0_\bquark}}\xspace}
\def\Xibm    {{\ensuremath{\Xires^-_\bquark}}\xspace}
\def\Xibbar  {{\ensuremath{\Xiresbar{}_\bquark}}\xspace}
\def\Xibbarz {{\ensuremath{\Xiresbar{}_\bquark^0}}\xspace}
\def\Xibbarp {{\ensuremath{\Xiresbar{}_\bquark^+}}\xspace}
\def\Xic     {{\ensuremath{\Xires_\cquark}}\xspace}
\def\Xicz    {{\ensuremath{\Xires^0_\cquark}}\xspace}
\def\Xicp    {{\ensuremath{\Xires^+_\cquark}}\xspace}
\def\Xicbar  {{\ensuremath{\Xiresbar{}_\cquark}}\xspace}
\def\Xicbarz {{\ensuremath{\Xiresbar{}_\cquark^0}}\xspace}
\def\Xicbarm {{\ensuremath{\Xiresbar{}_\cquark^-}}\xspace}
\def\Omegac    {{\ensuremath{\Omegares^0_\cquark}}\xspace}
\def\Omegacbar {{\ensuremath{\Omegaresbar{}_\cquark^0}}\xspace}
\def\Omegab    {{\ensuremath{\Omegares^-_\bquark}}\xspace}
\def\Omegabbar {{\ensuremath{\Omegaresbar{}_\bquark^+}}\xspace}

%%%%%%%%%%%%%%%%%%
% Physics symbols
%%%%%%%%%%%%%%%%%

%% Decays
\def\BF         {{\ensuremath{\mathcal{B}}}\xspace}
\def\BRvis      {{\ensuremath{\BR_{\mathrm{{vis}}}}}}
\def\BR         {\BF}
\newcommand{\decay}[2]{\ensuremath{#1\!\to #2}\xspace}         % {\Pa}{\Pb \Pc}
\def\ra                 {\ensuremath{\rightarrow}\xspace}
\def\to                 {\ensuremath{\rightarrow}\xspace}

%% Lifetimes
\newcommand{\tauBs}{{\ensuremath{\tau_{\Bs}}}\xspace}
\newcommand{\tauBd}{{\ensuremath{\tau_{\Bd}}}\xspace}
\newcommand{\tauBz}{{\ensuremath{\tau_{\Bz}}}\xspace}
\newcommand{\tauBu}{{\ensuremath{\tau_{\Bp}}}\xspace}
\newcommand{\tauDp}{{\ensuremath{\tau_{\Dp}}}\xspace}
\newcommand{\tauDz}{{\ensuremath{\tau_{\Dz}}}\xspace}
\newcommand{\tauL}{{\ensuremath{\tau_{\mathrm{ L}}}}\xspace}
\newcommand{\tauH}{{\ensuremath{\tau_{\mathrm{ H}}}}\xspace}

%% Masses
\newcommand{\mBd}{{\ensuremath{m_{\Bd}}}\xspace}
\newcommand{\mBp}{{\ensuremath{m_{\Bp}}}\xspace}
\newcommand{\mBs}{{\ensuremath{m_{\Bs}}}\xspace}
\newcommand{\mBc}{{\ensuremath{m_{\Bc}}}\xspace}
\newcommand{\mLb}{{\ensuremath{m_{\Lb}}}\xspace}

%% EW theory, groups
\def\grpsuthree {{\ensuremath{\mathrm{SU}(3)}}\xspace}
\def\grpsutw    {{\ensuremath{\mathrm{SU}(2)}}\xspace}
\def\grpuone    {{\ensuremath{\mathrm{U}(1)}}\xspace}

\def\ssqtw   {{\ensuremath{\sin^{2}\!\theta_{\mathrm{W}}}}\xspace}
\def\csqtw   {{\ensuremath{\cos^{2}\!\theta_{\mathrm{W}}}}\xspace}
\def\stw     {{\ensuremath{\sin\theta_{\mathrm{W}}}}\xspace}
\def\ctw     {{\ensuremath{\cos\theta_{\mathrm{W}}}}\xspace}
\def\ssqtwef {{\ensuremath{{\sin}^{2}\theta_{\mathrm{W}}^{\mathrm{eff}}}}\xspace}
\def\csqtwef {{\ensuremath{{\cos}^{2}\theta_{\mathrm{W}}^{\mathrm{eff}}}}\xspace}
\def\stwef   {{\ensuremath{\sin\theta_{\mathrm{W}}^{\mathrm{eff}}}}\xspace}
\def\ctwef   {{\ensuremath{\cos\theta_{\mathrm{W}}^{\mathrm{eff}}}}\xspace}
\def\gv      {{\ensuremath{g_{\mbox{\tiny V}}}}\xspace}
\def\ga      {{\ensuremath{g_{\mbox{\tiny A}}}}\xspace}

\def\order   {{\ensuremath{\mathcal{O}}}\xspace}
\def\ordalph {{\ensuremath{\mathcal{O}(\alpha)}}\xspace}
\def\ordalsq {{\ensuremath{\mathcal{O}(\alpha^{2})}}\xspace}
\def\ordalcb {{\ensuremath{\mathcal{O}(\alpha^{3})}}\xspace}

%% QCD parameters
\newcommand{\as}{{\ensuremath{\alpha_s}}\xspace}
\newcommand{\MSb}{{\ensuremath{\overline{\mathrm{MS}}}}\xspace}
\newcommand{\lqcd}{{\ensuremath{\Lambda_{\mathrm{QCD}}}}\xspace}
\def\qsq       {{\ensuremath{q^2}}\xspace}

%% CKM, CP violation

\def\eps   {{\ensuremath{\varepsilon}}\xspace}
\def\epsK  {{\ensuremath{\varepsilon_K}}\xspace}
\def\epsB  {{\ensuremath{\varepsilon_B}}\xspace}
\def\epsp  {{\ensuremath{\varepsilon^\prime_K}}\xspace}

\def\CP                {{\ensuremath{C\!P}}\xspace}
\def\CPT               {{\ensuremath{C\!PT}}\xspace}

\def\rhobar {{\ensuremath{\overline \rho}}\xspace}
\def\etabar {{\ensuremath{\overline \eta}}\xspace}

\def\Vud  {{\ensuremath{V_{\uquark\dquark}}}\xspace}
\def\Vcd  {{\ensuremath{V_{\cquark\dquark}}}\xspace}
\def\Vtd  {{\ensuremath{V_{\tquark\dquark}}}\xspace}
\def\Vus  {{\ensuremath{V_{\uquark\squark}}}\xspace}
\def\Vcs  {{\ensuremath{V_{\cquark\squark}}}\xspace}
\def\Vts  {{\ensuremath{V_{\tquark\squark}}}\xspace}
\def\Vub  {{\ensuremath{V_{\uquark\bquark}}}\xspace}
\def\Vcb  {{\ensuremath{V_{\cquark\bquark}}}\xspace}
\def\Vtb  {{\ensuremath{V_{\tquark\bquark}}}\xspace}
\def\Vuds  {{\ensuremath{V_{\uquark\dquark}^\ast}}\xspace}
\def\Vcds  {{\ensuremath{V_{\cquark\dquark}^\ast}}\xspace}
\def\Vtds  {{\ensuremath{V_{\tquark\dquark}^\ast}}\xspace}
\def\Vuss  {{\ensuremath{V_{\uquark\squark}^\ast}}\xspace}
\def\Vcss  {{\ensuremath{V_{\cquark\squark}^\ast}}\xspace}
\def\Vtss  {{\ensuremath{V_{\tquark\squark}^\ast}}\xspace}
\def\Vubs  {{\ensuremath{V_{\uquark\bquark}^\ast}}\xspace}
\def\Vcbs  {{\ensuremath{V_{\cquark\bquark}^\ast}}\xspace}
\def\Vtbs  {{\ensuremath{V_{\tquark\bquark}^\ast}}\xspace}

%% Oscillations

\newcommand{\dm}{{\ensuremath{\Delta m}}\xspace}
\newcommand{\dms}{{\ensuremath{\Delta m_{\squark}}}\xspace}
\newcommand{\dmd}{{\ensuremath{\Delta m_{\dquark}}}\xspace}
\newcommand{\DG}{{\ensuremath{\Delta\Gamma}}\xspace}
\newcommand{\DGs}{{\ensuremath{\Delta\Gamma_{\squark}}}\xspace}
\newcommand{\DGd}{{\ensuremath{\Delta\Gamma_{\dquark}}}\xspace}
\newcommand{\Gs}{{\ensuremath{\Gamma_{\squark}}}\xspace}
\newcommand{\Gd}{{\ensuremath{\Gamma_{\dquark}}}\xspace}
\newcommand{\MBq}{{\ensuremath{M_{\B_\quark}}}\xspace}
\newcommand{\DGq}{{\ensuremath{\Delta\Gamma_{\quark}}}\xspace}
\newcommand{\Gq}{{\ensuremath{\Gamma_{\quark}}}\xspace}
\newcommand{\dmq}{{\ensuremath{\Delta m_{\quark}}}\xspace}
\newcommand{\GL}{{\ensuremath{\Gamma_{\mathrm{ L}}}}\xspace}
\newcommand{\GH}{{\ensuremath{\Gamma_{\mathrm{ H}}}}\xspace}
\newcommand{\DGsGs}{{\ensuremath{\Delta\Gamma_{\squark}/\Gamma_{\squark}}}\xspace}
\newcommand{\Delm}{{\mbox{$\Delta m $}}\xspace}
\newcommand{\ACP}{{\ensuremath{{\mathcal{A}}^{\CP}}}\xspace}
\newcommand{\Adir}{{\ensuremath{{\mathcal{A}}^{\mathrm{ dir}}}}\xspace}
\newcommand{\Amix}{{\ensuremath{{\mathcal{A}}^{\mathrm{ mix}}}}\xspace}
\newcommand{\ADelta}{{\ensuremath{{\mathcal{A}}^\Delta}}\xspace}
\newcommand{\phid}{{\ensuremath{\phi_{\dquark}}}\xspace}
\newcommand{\sinphid}{{\ensuremath{\sin\!\phid}}\xspace}
\newcommand{\phis}{{\ensuremath{\phi_{\squark}}}\xspace}
\newcommand{\betas}{{\ensuremath{\beta_{\squark}}}\xspace}
\newcommand{\sbetas}{{\ensuremath{\sigma(\beta_{\squark})}}\xspace}
\newcommand{\stbetas}{{\ensuremath{\sigma(2\beta_{\squark})}}\xspace}
\newcommand{\stphis}{{\ensuremath{\sigma(\phi_{\squark})}}\xspace}
\newcommand{\sinphis}{{\ensuremath{\sin\!\phis}}\xspace}

%% Tagging
\newcommand{\edet}{{\ensuremath{\varepsilon_{\mathrm{ det}}}}\xspace}
\newcommand{\erec}{{\ensuremath{\varepsilon_{\mathrm{ rec/det}}}}\xspace}
\newcommand{\esel}{{\ensuremath{\varepsilon_{\mathrm{ sel/rec}}}}\xspace}
\newcommand{\etrg}{{\ensuremath{\varepsilon_{\mathrm{ trg/sel}}}}\xspace}
\newcommand{\etot}{{\ensuremath{\varepsilon_{\mathrm{ tot}}}}\xspace}

\newcommand{\mistag}{\ensuremath{\omega}\xspace}
\newcommand{\wcomb}{\ensuremath{\omega^{\mathrm{comb}}}\xspace}
\newcommand{\etag}{{\ensuremath{\varepsilon_{\mathrm{tag}}}}\xspace}
\newcommand{\etagcomb}{{\ensuremath{\varepsilon_{\mathrm{tag}}^{\mathrm{comb}}}}\xspace}
\newcommand{\effeff}{\ensuremath{\varepsilon_{\mathrm{eff}}}\xspace}
\newcommand{\effeffcomb}{\ensuremath{\varepsilon_{\mathrm{eff}}^{\mathrm{comb}}}\xspace}
\newcommand{\efftag}{{\ensuremath{\etag(1-2\omega)^2}}\xspace}
\newcommand{\effD}{{\ensuremath{\etag D^2}}\xspace}

\newcommand{\etagprompt}{{\ensuremath{\varepsilon_{\mathrm{ tag}}^{\mathrm{Pr}}}}\xspace}
\newcommand{\etagLL}{{\ensuremath{\varepsilon_{\mathrm{ tag}}^{\mathrm{LL}}}}\xspace}

%% Key decay channels

\def\BdToKstmm    {\decay{\Bd}{\Kstarz\mup\mun}}
\def\BdbToKstmm   {\decay{\Bdb}{\Kstarzb\mup\mun}}

\def\BsToJPsiPhi  {\decay{\Bs}{\jpsi\phi}}
\def\BdToJPsiKst  {\decay{\Bd}{\jpsi\Kstarz}}
\def\BdbToJPsiKst {\decay{\Bdb}{\jpsi\Kstarzb}}

\def\BsPhiGam     {\decay{\Bs}{\phi \g}}
\def\BdKstGam     {\decay{\Bd}{\Kstarz \g}}

\def\BTohh        {\decay{\B}{\Ph^+ \Ph'^-}}
\def\BdTopipi     {\decay{\Bd}{\pip\pim}}
\def\BdToKpi      {\decay{\Bd}{\Kp\pim}}
\def\BsToKK       {\decay{\Bs}{\Kp\Km}}
\def\BsTopiK      {\decay{\Bs}{\pip\Km}}

%% Rare decays
\def\BdKstee  {\decay{\Bd}{\Kstarz\epem}}
\def\BdbKstee {\decay{\Bdb}{\Kstarzb\epem}}
\def\bsll     {\decay{\bquark}{\squark \ell^+ \ell^-}}
\def\AFB      {\ensuremath{A_{\mathrm{FB}}}\xspace}
\def\FL       {\ensuremath{F_{\mathrm{L}}}\xspace}
\def\AT#1     {\ensuremath{A_{\mathrm{T}}^{#1}}\xspace}           % 2
\def\btosgam  {\decay{\bquark}{\squark \g}}
\def\btodgam  {\decay{\bquark}{\dquark \g}}
\def\Bsmm     {\decay{\Bs}{\mup\mun}}
\def\Bdmm     {\decay{\Bd}{\mup\mun}}
\def\ctl       {\ensuremath{\cos{\theta_\ell}}\xspace}
\def\ctk       {\ensuremath{\cos{\theta_K}}\xspace}

%% Wilson coefficients and operators
\def\C#1      {\ensuremath{\mathcal{C}_{#1}}\xspace}                       % 9
\def\Cp#1     {\ensuremath{\mathcal{C}_{#1}^{'}}\xspace}                    % 7
\def\Ceff#1   {\ensuremath{\mathcal{C}_{#1}^{\mathrm{(eff)}}}\xspace}        % 9  
\def\Cpeff#1  {\ensuremath{\mathcal{C}_{#1}^{'\mathrm{(eff)}}}\xspace}       % 7
\def\Ope#1    {\ensuremath{\mathcal{O}_{#1}}\xspace}                       % 2
\def\Opep#1   {\ensuremath{\mathcal{O}_{#1}^{'}}\xspace}                    % 7

%% Charm

\def\xprime     {\ensuremath{x^{\prime}}\xspace}
\def\yprime     {\ensuremath{y^{\prime}}\xspace}
\def\ycp        {\ensuremath{y_{\CP}}\xspace}
\def\agamma     {\ensuremath{A_{\Gamma}}\xspace}
%%%\def\kpi        {\ensuremath{\PK\Ppi}\xspace}
%%%\def\kk         {\ensuremath{\PK\PK}\xspace}
%%%\def\dkpi       {\decay{\PD}{\PK\Ppi}}
%%%\def\dkk        {\decay{\PD}{\PK\PK}}
\def\dkpicf     {\decay{\Dz}{\Km\pip}}

%% QM
\newcommand{\bra}[1]{\ensuremath{\langle #1|}}             % {a}
\newcommand{\ket}[1]{\ensuremath{|#1\rangle}}              % {b}
\newcommand{\braket}[2]{\ensuremath{\langle #1|#2\rangle}} % {a}{b}

%%%%%%%%%%%%%%%%%%%%%%%%%%%%%%%%%%%%%%%%%%%%%%%%%%
% Units
%%%%%%%%%%%%%%%%%%%%%%%%%%%%%%%%%%%%%%%%%%%%%%%%%%
\newcommand{\unit}[1]{\ensuremath{\mathrm{ \,#1}}\xspace}          % {kg}

%% Energy and momentum
\newcommand{\tev}{\ifthenelse{\boolean{inbibliography}}{\ensuremath{~T\kern -0.05em eV}\xspace}{\ensuremath{\mathrm{\,Te\kern -0.1em V}}}\xspace}
\newcommand{\gev}{\ensuremath{\mathrm{\,Ge\kern -0.1em V}}\xspace}
\newcommand{\mev}{\ensuremath{\mathrm{\,Me\kern -0.1em V}}\xspace}
\newcommand{\kev}{\ensuremath{\mathrm{\,ke\kern -0.1em V}}\xspace}
\newcommand{\ev}{\ensuremath{\mathrm{\,e\kern -0.1em V}}\xspace}
\newcommand{\gevc}{\ensuremath{{\mathrm{\,Ge\kern -0.1em V\!/}c}}\xspace}
\newcommand{\mevc}{\ensuremath{{\mathrm{\,Me\kern -0.1em V\!/}c}}\xspace}
\newcommand{\gevcc}{\ensuremath{{\mathrm{\,Ge\kern -0.1em V\!/}c^2}}\xspace}
\newcommand{\gevgevcccc}{\ensuremath{{\mathrm{\,Ge\kern -0.1em V^2\!/}c^4}}\xspace}
\newcommand{\mevcc}{\ensuremath{{\mathrm{\,Me\kern -0.1em V\!/}c^2}}\xspace}

%% Distance and area
\def\km   {\ensuremath{\mathrm{ \,km}}\xspace}
\def\m    {\ensuremath{\mathrm{ \,m}}\xspace}
\def\ma   {\ensuremath{{\mathrm{ \,m}}^2}\xspace}
\def\cm   {\ensuremath{\mathrm{ \,cm}}\xspace}
\def\cma  {\ensuremath{{\mathrm{ \,cm}}^2}\xspace}
\def\mm   {\ensuremath{\mathrm{ \,mm}}\xspace}
\def\mma  {\ensuremath{{\mathrm{ \,mm}}^2}\xspace}
\def\mum  {\ensuremath{{\,\upmu\mathrm{m}}}\xspace}
\def\muma {\ensuremath{{\,\upmu\mathrm{m}^2}}\xspace}
\def\nm   {\ensuremath{\mathrm{ \,nm}}\xspace}
\def\fm   {\ensuremath{\mathrm{ \,fm}}\xspace}
\def\barn{\ensuremath{\mathrm{ \,b}}\xspace}
%%%\def\barnhyph{\ensuremath{\mathrm{ -b}}\xspace}
\def\mbarn{\ensuremath{\mathrm{ \,mb}}\xspace}
\def\mub{\ensuremath{{\mathrm{ \,\upmu b}}}\xspace}
%%%\def\mbarnhyph{\ensuremath{\mathrm{ -mb}}\xspace}
\def\nb {\ensuremath{\mathrm{ \,nb}}\xspace}
\def\invnb {\ensuremath{\mbox{\,nb}^{-1}}\xspace}
\def\pb {\ensuremath{\mathrm{ \,pb}}\xspace}
\def\invpb {\ensuremath{\mbox{\,pb}^{-1}}\xspace}
\def\fb   {\ensuremath{\mbox{\,fb}}\xspace}
\def\invfb   {\ensuremath{\mbox{\,fb}^{-1}}\xspace}

%% Time 
\def\sec  {\ensuremath{\mathrm{{\,s}}}\xspace}
\def\ms   {\ensuremath{{\mathrm{ \,ms}}}\xspace}
\def\mus  {\ensuremath{{\,\upmu{\mathrm{ s}}}}\xspace}
\def\ns   {\ensuremath{{\mathrm{ \,ns}}}\xspace}
\def\ps   {\ensuremath{{\mathrm{ \,ps}}}\xspace}
\def\fs   {\ensuremath{\mathrm{ \,fs}}\xspace}

\def\mhz  {\ensuremath{{\mathrm{ \,MHz}}}\xspace}
\def\khz  {\ensuremath{{\mathrm{ \,kHz}}}\xspace}
\def\hz   {\ensuremath{{\mathrm{ \,Hz}}}\xspace}

\def\invps{\ensuremath{{\mathrm{ \,ps^{-1}}}}\xspace}
\def\invns{\ensuremath{{\mathrm{ \,ns^{-1}}}}\xspace}

\def\yr   {\ensuremath{\mathrm{ \,yr}}\xspace}
\def\hr   {\ensuremath{\mathrm{ \,hr}}\xspace}

%% Temperature
\def\degc {\ensuremath{^\circ}{C}\xspace}
\def\degk {\ensuremath {\mathrm{ K}}\xspace}

%% Material lengths, radiation
\def\Xrad {\ensuremath{X_0}\xspace}
\def\NIL{\ensuremath{\lambda_{int}}\xspace}
\def\mip {MIP\xspace}
\def\neutroneq {\ensuremath{\mathrm{ \,n_{eq}}}\xspace}
\def\neqcmcm {\ensuremath{\mathrm{ \,n_{eq} / cm^2}}\xspace}
\def\kRad {\ensuremath{\mathrm{ \,kRad}}\xspace}
\def\MRad {\ensuremath{\mathrm{ \,MRad}}\xspace}
\def\ci {\ensuremath{\mathrm{ \,Ci}}\xspace}
\def\mci {\ensuremath{\mathrm{ \,mCi}}\xspace}

%% Uncertainties
\def\sx    {\ensuremath{\sigma_x}\xspace}    
\def\sy    {\ensuremath{\sigma_y}\xspace}   
\def\sz    {\ensuremath{\sigma_z}\xspace}    

\newcommand{\stat}{\ensuremath{\mathrm{\,(stat)}}\xspace}
\newcommand{\syst}{\ensuremath{\mathrm{\,(syst)}}\xspace}

%% Maths

\def\order{{\ensuremath{\mathcal{O}}}\xspace}
\newcommand{\chisq}{\ensuremath{\chi^2}\xspace}
\newcommand{\chisqndf}{\ensuremath{\chi^2/\mathrm{ndf}}\xspace}
\newcommand{\chisqip}{\ensuremath{\chi^2_{\text{IP}}}\xspace}
\newcommand{\chisqvs}{\ensuremath{\chi^2_{\text{VS}}}\xspace}
\newcommand{\chisqvtx}{\ensuremath{\chi^2_{\text{vtx}}}\xspace}
\newcommand{\chisqvtxndf}{\ensuremath{\chi^2_{\text{vtx}}/\mathrm{ndf}}\xspace}

\def\deriv {\ensuremath{\mathrm{d}}}

\def\gsim{{~\raise.15em\hbox{$>$}\kern-.85em
          \lower.35em\hbox{$\sim$}~}\xspace}
\def\lsim{{~\raise.15em\hbox{$<$}\kern-.85em
          \lower.35em\hbox{$\sim$}~}\xspace}

\newcommand{\mean}[1]{\ensuremath{\left\langle #1 \right\rangle}} % {x}
\newcommand{\abs}[1]{\ensuremath{\left\|#1\right\|}} % {x}
\newcommand{\Real}{\ensuremath{\mathcal{R}e}\xspace}
\newcommand{\Imag}{\ensuremath{\mathcal{I}m}\xspace}

\def\PDF {PDF\xspace}

\def\sPlot{\mbox{\em sPlot}\xspace}
%%%\def\sWeight{\mbox{\em sWeight}\xspace}

%%%%%%%%%%%%%%%%%%%%%%%%%%%%%%%%%%%%%%%%%%%%%%%%%%
% Kinematics
%%%%%%%%%%%%%%%%%%%%%%%%%%%%%%%%%%%%%%%%%%%%%%%%%%

%% Energy, Momenta
\def\Ebeam {\ensuremath{E_{\mbox{\tiny BEAM}}}\xspace}
\def\sqs   {\ensuremath{\protect\sqrt{s}}\xspace}

\def\ptot       {\mbox{$p$}\xspace}
\def\pt         {\mbox{$p_{\mathrm{ T}}$}\xspace}
\def\et         {\mbox{$E_{\mathrm{ T}}$}\xspace}
\def\mt         {\mbox{$M_{\mathrm{ T}}$}\xspace}
\def\dpp        {\ensuremath{\Delta p/p}\xspace}
\def\msq        {\ensuremath{m^2}\xspace}
\newcommand{\dedx}{\ensuremath{\mathrm{d}\hspace{-0.1em}E/\mathrm{d}x}\xspace}

%% PID

\def\dllkpi     {\ensuremath{\mathrm{DLL}_{\kaon\pion}}\xspace}
\def\dllppi     {\ensuremath{\mathrm{DLL}_{\proton\pion}}\xspace}
\def\dllepi     {\ensuremath{\mathrm{DLL}_{\electron\pion}}\xspace}
\def\dllmupi    {\ensuremath{\mathrm{DLL}_{\muon\pi}}\xspace}

%% Geometry
%%%\def\mphi       {\mbox{$\phi$}\xspace}
%%%\def\mtheta     {\mbox{$\theta$}\xspace}
%%%\def\ctheta     {\mbox{$\cos\theta$}\xspace}
%%%\def\stheta     {\mbox{$\sin\theta$}\xspace}
%%%\def\ttheta     {\mbox{$\tan\theta$}\xspace}

\def\degrees{\ensuremath{^{\circ}}\xspace}
\def\krad {\ensuremath{\mathrm{ \,krad}}\xspace}
\def\mrad{\ensuremath{\mathrm{ \,mrad}}\xspace}
\def\rad{\ensuremath{\mathrm{ \,rad}}\xspace}

%% Accelerator
\def\betastar {\ensuremath{\beta^*}}
\newcommand{\lum} {\ensuremath{\mathcal{L}}\xspace}
\newcommand{\intlum}[1]{\ensuremath{\int\lum=#1}\xspace}  % {2 \,\invfb}

%%%%%%%%%%%%%%%%%%%%%%%%%%%%%%%%%%%%%%%%%%%%%%%%%%%%%%%%%%%%%%%%%%%%
% Software
%%%%%%%%%%%%%%%%%%%%%%%%%%%%%%%%%%%%%%%%%%%%%%%%%%%%%%%%%%%%%%%%%%%%

%% Programs
%%%\def\ansys      {\mbox{\textsc{Ansys}}\xspace}
\def\bcvegpy    {\mbox{\textsc{Bcvegpy}}\xspace}
\def\boole      {\mbox{\textsc{Boole}}\xspace}
\def\brunel     {\mbox{\textsc{Brunel}}\xspace}
\def\davinci    {\mbox{\textsc{DaVinci}}\xspace}
\def\dirac      {\mbox{\textsc{Dirac}}\xspace}
%%%\def\erasmus    {\mbox{\textsc{Erasmus}}\xspace}
\def\evtgen     {\mbox{\textsc{EvtGen}}\xspace}
\def\fewz       {\mbox{\textsc{Fewz}}\xspace}
\def\fluka      {\mbox{\textsc{Fluka}}\xspace}
\def\ganga      {\mbox{\textsc{Ganga}}\xspace}
%%%\def\garfield   {\mbox{\textsc{Garfield}}\xspace}
\def\gaudi      {\mbox{\textsc{Gaudi}}\xspace}
\def\gauss      {\mbox{\textsc{Gauss}}\xspace}
\def\geant      {\mbox{\textsc{Geant4}}\xspace}
\def\hepmc      {\mbox{\textsc{HepMC}}\xspace}
\def\herwig     {\mbox{\textsc{Herwig}}\xspace}
\def\moore      {\mbox{\textsc{Moore}}\xspace}
\def\neurobayes {\mbox{\textsc{NeuroBayes}}\xspace}
\def\photos     {\mbox{\textsc{Photos}}\xspace}
\def\powheg     {\mbox{\textsc{Powheg}}\xspace}
%%%\def\pyroot     {\mbox{\textsc{PyRoot}}\xspace}
\def\pythia     {\mbox{\textsc{Pythia}}\xspace}
\def\resbos     {\mbox{\textsc{ResBos}}\xspace}
\def\roofit     {\mbox{\textsc{RooFit}}\xspace}
\def\root       {\mbox{\textsc{Root}}\xspace}
\def\spice      {\mbox{\textsc{Spice}}\xspace}
%%%\def\tosca      {\mbox{\textsc{Tosca}}\xspace}
\def\urania     {\mbox{\textsc{Urania}}\xspace}

%% Languages
\def\cpp        {\mbox{\textsc{C\raisebox{0.1em}{{\footnotesize{++}}}}}\xspace}
%%%\def\python     {\mbox{\textsc{Python}}\xspace}
\def\ruby       {\mbox{\textsc{Ruby}}\xspace}
\def\fortran    {\mbox{\textsc{Fortran}}\xspace}
\def\svn        {\mbox{\textsc{SVN}}\xspace}

%% Data processing
\def\kbytes     {\ensuremath{{\mathrm{ \,kbytes}}}\xspace}
\def\kbsps      {\ensuremath{{\mathrm{ \,kbytes/s}}}\xspace}
\def\kbits      {\ensuremath{{\mathrm{ \,kbits}}}\xspace}
\def\kbsps      {\ensuremath{{\mathrm{ \,kbits/s}}}\xspace}
\def\mbsps      {\ensuremath{{\mathrm{ \,Mbits/s}}}\xspace}
\def\mbytes     {\ensuremath{{\mathrm{ \,Mbytes}}}\xspace}
\def\mbps       {\ensuremath{{\mathrm{ \,Mbyte/s}}}\xspace}
\def\mbsps      {\ensuremath{{\mathrm{ \,Mbytes/s}}}\xspace}
\def\gbsps      {\ensuremath{{\mathrm{ \,Gbits/s}}}\xspace}
\def\gbytes     {\ensuremath{{\mathrm{ \,Gbytes}}}\xspace}
\def\gbsps      {\ensuremath{{\mathrm{ \,Gbytes/s}}}\xspace}
\def\tbytes     {\ensuremath{{\mathrm{ \,Tbytes}}}\xspace}
\def\tbpy       {\ensuremath{{\mathrm{ \,Tbytes/yr}}}\xspace}

\def\dst        {DST\xspace}

%%%%%%%%%%%%%%%%%%%%%%%%%%%
% Detector related
%%%%%%%%%%%%%%%%%%%%%%%%%%%

%% Detector technologies
\def\nonn {\ensuremath{\mathrm{{ \mathit{n^+}} \mbox{-} on\mbox{-}{ \mathit{n}}}}\xspace}
\def\ponn {\ensuremath{\mathrm{{ \mathit{p^+}} \mbox{-} on\mbox{-}{ \mathit{n}}}}\xspace}
\def\nonp {\ensuremath{\mathrm{{ \mathit{n^+}} \mbox{-} on\mbox{-}{ \mathit{p}}}}\xspace}
\def\cvd  {CVD\xspace}
\def\mwpc {MWPC\xspace}
\def\gem  {GEM\xspace}

%% Detector components, electronics
\def\tell1  {TELL1\xspace}
\def\ukl1   {UKL1\xspace}
\def\beetle {Beetle\xspace}
\def\otis   {OTIS\xspace}
\def\croc   {CROC\xspace}
\def\carioca {CARIOCA\xspace}
\def\dialog {DIALOG\xspace}
\def\sync   {SYNC\xspace}
\def\cardiac {CARDIAC\xspace}
\def\gol    {GOL\xspace}
\def\vcsel  {VCSEL\xspace}
\def\ttc    {TTC\xspace}
\def\ttcrx  {TTCrx\xspace}
\def\hpd    {HPD\xspace}
\def\pmt    {PMT\xspace}
\def\specs  {SPECS\xspace}
\def\elmb   {ELMB\xspace}
\def\fpga   {FPGA\xspace}
\def\plc    {PLC\xspace}
\def\rasnik {RASNIK\xspace}
\def\elmb   {ELMB\xspace}
\def\can    {CAN\xspace}
\def\lvds   {LVDS\xspace}
\def\ntc    {NTC\xspace}
\def\adc    {ADC\xspace}
\def\led    {LED\xspace}
\def\ccd    {CCD\xspace}
\def\hv     {HV\xspace}
\def\lv     {LV\xspace}
\def\pvss   {PVSS\xspace}
\def\cmos   {CMOS\xspace}
\def\fifo   {FIFO\xspace}
\def\ccpc   {CCPC\xspace}

%% Chemical symbols
\def\cfourften     {\ensuremath{\mathrm{ C_4 F_{10}}}\xspace}
\def\cffour        {\ensuremath{\mathrm{ CF_4}}\xspace}
\def\cotwo         {\ensuremath{\mathrm{ CO_2}}\xspace} 
\def\csixffouteen  {\ensuremath{\mathrm{ C_6 F_{14}}}\xspace} 
\def\mgftwo     {\ensuremath{\mathrm{ Mg F_2}}\xspace} 
\def\siotwo     {\ensuremath{\mathrm{ SiO_2}}\xspace} 

%%%%%%%%%%%%%%%
% Special Text 
%%%%%%%%%%%%%%%
\newcommand{\eg}{\mbox{\itshape e.g.}\xspace}
\newcommand{\ie}{\mbox{\itshape i.e.}\xspace}
\newcommand{\etal}{\mbox{\itshape et al.}\xspace}
\newcommand{\etc}{\mbox{\itshape etc.}\xspace}
\newcommand{\cf}{\mbox{\itshape cf.}\xspace}
\newcommand{\ffp}{\mbox{\itshape ff.}\xspace}
\newcommand{\vs}{\mbox{\itshape vs.}\xspace}
 % Add in the predefined LHCb symbols
%
%% Make this the last packages you include before the \begin{document}
%\usepackage{cite} % Allows for ranges in citations
%\usepackage{mciteplus}
%
\usepackage{enumerate}
